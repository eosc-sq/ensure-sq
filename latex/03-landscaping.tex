\subsection{Definition and references of Research Software}
\label{subsec:defrs}

Research software are not only software. To be able to clearly
identify quality attributes for research software, it is essential to
define the object we are talking about.

The Source Codes and Softwares College from the Committee for Open
Science in France proposes the following definition: 
``Research software is developed to meet specific scientific needs. It
is designed, maintained, and used by scientists (researchers and
engineers) and research institutions, possibly on an international
scale. It can result from research work as well as support it, notably
through publications before/on/around/with the software. It can be
formalized in different ways (a platform, a middleware, a workflow or
a library, a module or plugin of another software) and thus be in
interaction in an ecosystem or on the contrary be more
autonomous.'' \footnote{(\url{https://www.ouvrirlascience.fr/research-software-as-a-pillar-of-open-science/}).}

The most important characteristic is the link with publications. The
purpose of research software is that it support the scientific
results, and from that it is expected to reproduce the published
results. Regular software would not necessarily have this expectation.

The goal of this section is to cross-reference the elements of
software quality from the previous section with the subgroup 1
\cite{sg1tf2023} ("Software Lifecycle", see section
\ref{sec:introduction}), typology of research software development, in
order to identify the tools and infrastructures needed to satisfy
minimum quality criteria, given the specificities of research software.  

As a reminder, subgroup 1 has defined different categories of software
development related to the context in which they are performed:

\begin{enumerate}
    \item \textbf{Individual development} - Individual creating research software for own use (e.g. a PhD student): Based on a research question, software is created by a single person with the specific aim of answering the research question and producing research output (paper, dataset, etc).
    \item \textbf{Team Project} - A research team creating an application or workflow for use within the team: Research software is created by a team to answer a series of research questions (often as part of a larger research project).
    \item \textbf{Team OSS} - A team / community developing (possibly broadly applicable) open source research software: Software is created by a team (possibly distributed over multiple organizations) to answer a broad range of research questions.
    \item \textbf{Team Service} - A team or community creating a research service: A service platform is a set of software components which is used to provide services for a large number of users, most of whom make use of these offerings via the Internet (e.g. cloud services).
\end{enumerate}

There are other proposals with different categorization, for instance \footnote{\url{https://docs.google.com/presentation/d/1uwxSwd8chbG7bVn5lPvNNhv5f_JeeelA2mN9NXhlOhA}} and \footnote{\url{https://zenodo.org/record/7589725}}.

\subsection{Research Software stack}

We can have another approach to categorize research software based on
their position in the global software stack, following
\cite{hinsen2019}.

\dg{Should comment on the Tom Honeyman proposal for research software: research code (i.e., glue), research prototypes (tools used for research) and research infrastructure. I think it makes sense for us too}

A typical scientific software stack~\cite{hinsen2019} has the following composition:

\begin{center}
    \bottomcaption{RS stacks, stacks definitions and RS types. Stack level 1 is not considered in this report.}
    \label{tab:rs_stacks}
    \small
    \begin{tabular}[t]{|p{0.15\linewidth}|p{0.5\linewidth}|p{0.25\linewidth}|} \hline

    \textbf{Stack} & \textbf{Stack definition} & \textbf{Types} \\ \hline \hline
      4 - Project specific code &
    Software written by scientists for a specific research project. It can take various forms including scripts, notebooks, and workflows, but also special-purpose libraries and utilities. &
    Library; Analysis script and workflows; Services and platforms \\ \hline

    3 - Domain specific tools &
    Tools and libraries that implement models and methods which are developed and used by specific communities. Gromacs, MMTK, Amber. &
    Library; Application; Services and platforms \\ \hline

    2 - Scientific infrastructure &
    Infrastructure created specifically for scientific computing, but not for any particular application domain: mathematical libraries such as BLAS, LAPACK, or SciPy, scientific data management tools such as HDF5. &
    Library; Framework; Services and platforms\\ \hline

    \it{1 - Non-scientific infrastructure} &
    \it{Compilers and interpreters, libraries for data management; gcc, python...} &
    -- \\ \hline

    \end{tabular}
\end{center}

Note that Stack level 1 is not considered in this report. The Research Software types are summarised next including some examples:

\begin{enumerate}
    \item \textbf{Library}: SciPy, TensorFlow, FFTW, BLAS, LAPACK.
    \item \textbf{Framework}: JupyterLab, RStudio.
    \item \textbf{Application} (such as Monte-Carlo simulation Software): Gromacs, Amber, GEANT4, COMSOL.
    \item \textbf{Analysis script and workflows}
    \item \textbf{Services and platforms}: Galaxy server, Scipion, SAPS, O3AS.
\end{enumerate}


\subsection{Expectation for Research Software stack and types}

To be able to identify quality attributes for research
software, we need to understand clearly what is expected for the
different types identified in the two above sections.

\subsubsection{By research software type}

The different expectations described below add up to each other, when
the context of development grow in complexity.

\textbf{Individual researcher creating software for personal use}

The author must be able to manage the different versions of the
code. He often has to come back to the software after a long time (to
answer to a reviewer for a publication, to adress a new scientific
question ...).

He perhaps needs to excute the software on different computers and
different operating systems (for example the use of a supercomputer).

He has to share the code when he submits his article.

\textbf{A research team creating an application or workflow for use
  within the team}

The team must be able to manage the development with several
people. The way to program must be homogeneous, and everybody must
have a clear vision of who does what.

New arrival in the team must be able to easily understand the
structure of the software and the way to develop in the code.

\textbf{A team / community developing open source software}

For a community research software, it is important to facilitate
different types of contribution and to manage correctly rights
attribution.

Taking into consideration validation and verification is crucial. The
code has to be easy to maintain et to upgrade.

A community software must be as easy as possible to install, use and report
bug, problems or suggestions.
A real animation around the software must be organized with support to users.

\textbf{A team or community creating a service, a platform or an
  infrastructure}

The main expectation is here the disponibility of the service and
insure security for users.

\textbf{A team / community developing software in an industrial
  context}

In the case of industrial context, expectations can vary depending of
the contract between academic researchers and compagny.

\subsubsection{Layer by layer in the software stack}

For this categorization, the point of view is more a user point of
view than a developer point of view.

\textbf{A scientific infrastructure}

A scientific infrastructure as described in \cite{hinsen2019} is by
definition a software used in another software. It is not a standalone
code.

It must therefore be easy to install, and easy to interface with
another software.
Users expect that this type of software are robust and with a
sufficient sustainability because it is often the foundation of their code.

\textbf{A domain specific tool}

A domain specific tool is close to a community software. It must be
easy to install on different operating systems, and easy to use. It
should have good retrocompatibility.

It must also be easy to cite. It is important for users to be able to
know precisely what the software does without having to read the source. 

\textbf{A project specific code}

A project specific code is close to a personal or a team code. It has
a important role
for reproductibility. It must be easily citable and archive for
sustainibility reasons.

It often depends of scientific infrastructures.
