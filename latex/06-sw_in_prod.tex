\subsection{Quality Attributes related to SW release management}

The following list are the Quality Attributes recommended for SW release and management:

\begin{itemize}
    \item \textbf{EOSC-TMet-02}: \# Resolved bugs
    \item \textbf{EOSC-TMet-03}: \# Open bugs
    \item \textbf{EOSC-TMet-04}: Defect rates
    \item \textbf{EOSC-SWRelMan-07}: Support
    \item \textbf{EOSC-SWRelMan-09}: Semantic Versioning
    \item \textbf{EOSC-SWRelMan-12}: Packaging
    \item \textbf{EOSC-SWRelMan-13}: Register/publish artifact
    \item \textbf{EOSC-SWRelMan-17}: SCM tool
    \item \textbf{EOSC-SWRelMan-20}: Installability
    \item \textbf{EOSC-SWRelMan-25}: Provide checksums
    \item \textbf{EOSC-SWRelMan-29}: Documentation online
    \item \textbf{EOSC-SWRelMan-32}: Documentation production
\end{itemize}

\subsection{Quality Attributes related to SW and services in production}

The following list are the Quality Attributes recommended for Services and platforms in production:

\begin{itemize}
    \item \textbf{EOSC-TMet-02}: \# Resolved bugs
    \item \textbf{EOSC-TMet-03}: \# Open bugs
    \item \textbf{EOSC-TMet-04}: Defect rates
    \item \textbf{EOSC-TMet-08}: \# Registered users
    \item \textbf{EOSC-TMet-09}: \# Active users
    \item \textbf{EOSC-TMet-10}: Amount computing resources
    \item \textbf{EOSC-TMet-11}: Amount storage resources
    \item \textbf{EOSC-SWRelMan-07}: Support
    \item \textbf{EOSC-SWRelMan-17}: SCM tool
    \item \textbf{EOSC-SWRelMan-20}: Installability
    \item \textbf{EOSC-SWRelMan-22}: Infrastructure as Code (IaC) validation
    \item \textbf{EOSC-SWRelMan-29}: Documentation online
    \item \textbf{EOSC-SWRelMan-32}: Documentation production
    \item \textbf{EOSC-SWTest-16}: Public endpoints and APIs encrypted
    \item \textbf{EOSC-SWTest-17}: Strong ciphers
    \item \textbf{EOSC-SWTest-18}: Authentication and Authorization
    \item \textbf{EOSC-SWTest-20}: Service compliance with data regulations (GDPR)
    \item \textbf{EOSC-SWTest-24}: Security assessment
    \item \textbf{EOSC-SrvOps-01}: Acceptable Usage Policy (AUP)
    \item \textbf{EOSC-SrvOps-02}: Access Policy and Terms of Use
    \item \textbf{EOSC-SrvOps-03}: Privacy policy
    \item \textbf{EOSC-SrvOps-04}: Operational Level Agreement (OLA)
    \item \textbf{EOSC-SrvOps-05}: Service Level Agreement (SLA)
    \item \textbf{EOSC-SrvOps-06}: Monitoring service public endpoints
    \item \textbf{EOSC-SrvOps-07}: Monitoring service public APIs
    \item \textbf{EOSC-SrvOps-09}: Monitoring security public endpoints and APIs
\end{itemize}

%%%%%%%%%%%%%%%%%%%%%%%%%%%%%%%%%%%%%%%%%%%%%%%%%%%%%%%%%
\newpage

\miguel{Scalability. Depending on the volume of the data or the computational needs, it's a matter of system design, but eventually of the
quality of the platform/software.}

\dg{[Jael, Daniel] Make sure this work aligns with SG1, which takes into account the software research cycle.}

\textbf{Abdulrahman} Define an agreed upon procedure for SW packaging. What to have as containers/conda packages, what to have as e.g. Easybuild recipes. Should we promote one portable solution or a combination of several?

\eli{Portability. Depending on the design of the software.}

\miguel{The problem of the containers (Docket et al.) and software development in online platforms. Advantage of containers: easier to release
and to ensure reproducibility and to ensure the service. Disadvantages: containers freeze the environment, and thus they stop receiving bug
fixing and security updates. Dependency with outdated software components.}

\miguel{Quality of documentation: is the documentation of the research SW always synchronized with its release?}

\miguel{Is the research SW associated with any scientific papers?
For example, k-means clustering in scikit-learn \url{https://scikit-learn.org/stable/modules/clustering.html\#k-means}) is associated with:
k-means++: The advantages of careful seeding” Arthur, David, and Sergei Vassilvitskii, Proceedings of the eighteenth annual ACM-SIAM symposium
on Discrete algorithms, Society for Industrial and Applied Mathematics (2007)
The source code provides the author names, but no reference paper:
\url{https://github.com/scikit-learn/scikit-learn/blob/main/sklearn/cluster/_kmeans.py}
However, is it associated with any other paper which provides a detailed procedure and a pseudocode? Without this, one can't have a good level of reproducibility.}

\miguel{Does the SW perform self tests? Say, unit tests.}

\miguel{Does the SW follow some known methodology which includes verifications in order to ensure it works well? At which level? For example,
if it uses continuous integration, does it include any step to ensure that the results are scientifically valid, or just unit tests?}

\miguel{Stability of research SW. Does it ensure backwards compatibility between versions?}

\cerlane{Do we mean 'Maintenance' or 'Maintainability'?}

\cerlane{Traditionally “portability” can refer to the portability across hardware architecture, system and platform (compilers and libraries)
by applying good software engineering practices. With containers, portability can also be achieved by restricting the supported system and platform
in the container. A multi-architecture container can further improve portability.}

\cerlane{Reproducibility is easier to achieve on the same system, software and hardware. However, it doesn't always imply a complete numerical
reproducibility. To achieve numerical reproducibility, one would have to even avoid the use of certain optimisation compiler options, optimised Maths 
libraries and apply very good software engineering techniques when applying algorithms, particularly with MPI.}

\mdavid{We will take the SW lifecycle diagram from SG1 and connect it here, or some of the previous sections.}
