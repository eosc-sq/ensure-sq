\subsection{Quality Attributes recommendations}
\label{subsec:attributesrecomen}

\subsubsection{Common recommendations}

Table~\ref{tab:rs_recc00} shows the list of Quality Attributes that are common to all types of Software, and to all user stories described in subsection~\ref{subsec:defrs}. They are deemed important and not too difficult to implement even by an individual researcher or software developer.

These attributes are about documentation, open source code including a license, visible description about intellectual property that is important to establish ownership and scientific recognition. They also include avoiding redundant code and addressing security concerns (e.g. no world writable files and directories).

\begin{table}[h]
  \centering
  \scriptsize
  \begin{tabular}{|p{0.55\linewidth}|p{0.3\linewidth}|} \hline

    \textbf{RS Stack} & \textbf{RS Type} \\ \hline \hline
    \textbf{All} & \textbf{All} \\ \hline
    \multicolumn{2}{|l|}{\textbf{EOSC-SCMet-02}: \% of redundant code} \\ \hline
    \multicolumn{2}{|l|}{\textbf{EOSC-SCMet-10}: Number of comments} \\ \hline
    \multicolumn{2}{|l|}{\textbf{EOSC-Qual-27}: Intellectual Property} \\ \hline
    \multicolumn{2}{|l|}{\textbf{EOSC-SWRelMan-01}: Open source} \\ \hline
    \multicolumn{2}{|l|}{\textbf{EOSC-SWRelMan-02}: Version Control System (VCS)} \\ \hline
    \multicolumn{2}{|l|}{\textbf{EOSC-SWRelMan-10}: Open-source license} \\ \hline
    \multicolumn{2}{|l|}{\textbf{EOSC-SWRelMan-29}: Documentation online} \\ \hline
    \multicolumn{2}{|l|}{\textbf{EOSC-SWRelMan-30}: Documentation updates} \\ \hline
    \multicolumn{2}{|l|}{\textbf{EOSC-SWRelMan-32}: Documentation production} \\ \hline
    \multicolumn{2}{|l|}{\textbf{EOSC-SWTest-15}: No world-writable files or directories} \\ \hline

  \end{tabular}
  \caption{Recommended common Quality Attributes.}
  \label{tab:rs_recc00}
\end{table}

\subsubsection{User story: Individual development}

In the case of individual development of analysis script and workflows (Table~\ref{tab:rs_recc01}), the recommendations are the common attributes listed in Table~\ref{tab:rs_recc00}.

\begin{table}[h]
  \centering
  \scriptsize
  \begin{tabular}{|p{0.55\linewidth}|p{0.3\linewidth}|} \hline

    \textbf{RS Stack} & \textbf{RS Type} \\ \hline \hline
    4: \textbf{Project specific code} & 4: \textbf{Analysis script and workflows} \\ \hline
    \multicolumn{2}{|l|}{\textbf{All common Quality Attributes}: Table \ref{tab:rs_recc00}} \\ \hline

  \end{tabular}
  \caption{Recommended Quality Attributes, individual development, Analysis script and workflows.}
  \label{tab:rs_recc01}
\end{table}

In the case of software type ``Library'', we add packaging and code deployment to the common attributes as listed in Table \ref{tab:rs_recc02} (c.f. RS lifecycle diagram \ref{fig:rslifecycle} box 5).

\begin{table}[h]
  \centering
  \scriptsize
  \begin{tabular}{|p{0.55\linewidth}|p{0.3\linewidth}|} \hline

    \textbf{RS Stack} & \textbf{RS Type} \\ \hline \hline
    4: \textbf{Project specific code} & 1: \textbf{Library} \\ \hline
    \multicolumn{2}{|l|}{\textbf{All common Quality Attributes}: Table \ref{tab:rs_recc00}} \\ \hline
    \multicolumn{2}{|l|}{\textbf{EOSC-SWRelMan-12}: Packaging} \\ \hline
    \multicolumn{2}{|l|}{\textbf{EOSC-SWRelMan-15}: Code deployment} \\ \hline

  \end{tabular}
  \caption{Recommended Quality Attributes, individual development, library.}
  \label{tab:rs_recc02}
\end{table}

\subsubsection{User story: Team Project}

In the case of a team working on a project on analysis script and workflows, there is an increase in responsibility. As such it's recommended to add support that includes the reporting and solving bugs and source code hosting in common VCS repositories, as shown in Table \ref{tab:rs_recc03}. See RS lifecycle diagram \ref{fig:rslifecycle} boxes \textit{Maintenance} and \textit{Community Feedback}.

\begin{table}[h]
  \centering
  \scriptsize
  \begin{tabular}{|p{0.55\linewidth}|p{0.3\linewidth}|} \hline

    \textbf{RS Stack} & \textbf{RS Type} \\ \hline \hline
    4: \textbf{Project specific code} & 4: \textbf{Analysis script and workflows} \\ \hline
    \multicolumn{2}{|l|}{\textbf{All from User story: Individual development}: Table \ref{tab:rs_recc01}} \\ \hline
    \multicolumn{2}{|l|}{\textbf{EOSC-TMet-02}: \# Resolved bugs} \\ \hline
    \multicolumn{2}{|l|}{\textbf{EOSC-TMet-03}: \# Open bugs} \\ \hline
    \multicolumn{2}{|l|}{\textbf{EOSC-SWRelMan-03}: Source code hosting} \\ \hline
    \multicolumn{2}{|l|}{\textbf{EOSC-SWRelMan-04}: Working state version} \\ \hline
    \multicolumn{2}{|l|}{\textbf{EOSC-SWRelMan-07}: Support} \\ \hline

  \end{tabular}
  \caption{Recommended Quality Attributes, team project, analysis script and workflows.}
  \label{tab:rs_recc03}
\end{table}

The case of Research Software type ``Library'', the recommendation is to add testing with respect to the previous cases, this is shown in table \ref{tab:rs_recc04}.
See RS lifecycle diagram \ref{fig:rslifecycle} boxes \textit{Maintenance}, \textit{Community Feedback} and \textit{Testing/CICD}.

\begin{table}[h]
  \centering
  \scriptsize
  \begin{tabular}{|p{0.55\linewidth}|p{0.3\linewidth}|} \hline

    \textbf{RS Stack} & \textbf{RS Type} \\ \hline \hline
    4: \textbf{Project specific code} & 1: \textbf{Library} \\ \hline
    \multicolumn{2}{|l|}{\textbf{All from User story: Individual development}: Table \ref{tab:rs_recc02}} \\ \hline
    \multicolumn{2}{|l|}{\textbf{EOSC-TMet-02}: \# Resolved bugs} \\ \hline
    \multicolumn{2}{|l|}{\textbf{EOSC-TMet-03}: \# Open bugs} \\ \hline
    \multicolumn{2}{|l|}{\textbf{EOSC-SWRelMan-03}: Source code hosting} \\ \hline
    \multicolumn{2}{|l|}{\textbf{EOSC-SWRelMan-04}: Working state version} \\ \hline
    \multicolumn{2}{|l|}{\textbf{EOSC-SWRelMan-07}: Support} \\ \hline
    \multicolumn{2}{|l|}{\textbf{EOSC-SWTest-01}: Code style} \\ \hline
    \multicolumn{2}{|l|}{\textbf{EOSC-SWTest-02}: Unit tests} \\ \hline
    \multicolumn{2}{|l|}{\textbf{EOSC-SWTest-03}: Test doubles} \\ \hline
    \multicolumn{2}{|l|}{\textbf{EOSC-SWTest-07}: Functional testing} \\ \hline

  \end{tabular}
  \caption{Recommended Quality Attributes, team project, library.}
  \label{tab:rs_recc04}
\end{table}

Table \ref{tab:rs_recc05} regards the development, deployment and operation of a service. We add recommendations related to further types of tests such as APIs and integration (when applicable). Furthermore, there are several Quality Attributes related to security such as the use of service certificates to encrypt endpoints, use of strong authentication and authorization mechanisms even in the case of limited access to the services. See RS lifecycle diagram \ref{fig:rslifecycle} boxes \textit{Maintenance}, \textit{Community Feedback} and \textit{Testing/CICD}.

\begin{table}[h]
  \centering
  \scriptsize
  \begin{tabular}{|p{0.55\linewidth}|p{0.3\linewidth}|} \hline

    \textbf{RS Stack} & \textbf{RS Type} \\ \hline \hline
    4: \textbf{Project specific code} & 5: \textbf{Services and platforms} \\ \hline
    \multicolumn{2}{|l|}{\textbf{All from User story: team project, library}: Table \ref{tab:rs_recc04}} \\ \hline
    \multicolumn{2}{|l|}{\textbf{EOSC-SWTest-05}: API testing} \\ \hline
    \multicolumn{2}{|l|}{\textbf{EOSC-SWTest-06}: Integration testing} \\ \hline
    \multicolumn{2}{|l|}{\textbf{EOSC-SWTest-16}: Public endpoints and APIs encrypted} \\ \hline
    \multicolumn{2}{|l|}{\textbf{EOSC-SWTest-17}: Strong ciphers} \\ \hline
    \multicolumn{2}{|l|}{\textbf{EOSC-SWTest-18}: Authentication and Authorization} \\ \hline
    \multicolumn{2}{|l|}{\textbf{EOSC-SWTest-20}: Service compliance with data regulations (GDPR)} \\ \hline

  \end{tabular}
  \caption{Recommended Quality Attributes, team project, services and platforms.}
  \label{tab:rs_recc05}
\end{table}

\subsubsection{User story: Team OSS}

This subsection describes the recommendations when a medium to large team is responsible for the development of Open Source Software for research. Thus, the increase in responsibility with respect to the previous cases of smaller teams or individual researchers. We add Quality Attributes for more complete documentation, additional types of tests including security and peer code review, as shown in Table \ref{tab:rs_recc06}. See RS lifecycle diagram \ref{fig:rslifecycle} box \textit{Testing/CICD}.

\begin{table}[h]
  \centering
  \scriptsize
  \begin{tabular}{|p{0.55\linewidth}|p{0.3\linewidth}|} \hline

    \textbf{RS Stack} & \textbf{RS Type} \\ \hline \hline
    3 and 2: \textbf{Domain specific tools} and  \textbf{Scientific infrastructure} & 1: \textbf{Library} \\ \hline
    \multicolumn{2}{|l|}{\textbf{All from User story: Team Project}: Table \ref{tab:rs_recc04}} \\ \hline
    \multicolumn{2}{|l|}{\textbf{EOSC-TMet-04}: Defect rates} \\ \hline
    \multicolumn{2}{|l|}{\textbf{EOSC-SWRelMan-08}: Code review} \\ \hline
    \multicolumn{2}{|l|}{\textbf{EOSC-SWRelMan-26}: Documentation version controlled} \\ \hline
    \multicolumn{2}{|l|}{\textbf{EOSC-SWRelMan-27}: Documentation as code} \\ \hline
    \multicolumn{2}{|l|}{\textbf{EOSC-SWRelMan-28}: Documentation formats} \\ \hline
    \multicolumn{2}{|l|}{\textbf{EOSC-SWTest-08}: Performance testing} \\ \hline
    \multicolumn{2}{|l|}{\textbf{EOSC-SWTest-13}: Static Application Security Testing (SAST)} \\ \hline
    \multicolumn{2}{|l|}{\textbf{EOSC-SWTest-14}: Security code reviews} \\ \hline

  \end{tabular}
  \caption{Recommended Quality Attributes, team OSS, library.}
  \label{tab:rs_recc06}
\end{table}

Table \ref{tab:rs_recc07} shows the case for a Framework or Application, it adds further types of testing such as scalability and development methodology called Test-Driven Development (TDD). See RS lifecycle diagram \ref{fig:rslifecycle} box \textit{Testing/CICD}.

\begin{table}[h]
  \centering
  \scriptsize
  \begin{tabular}{|p{0.55\linewidth}|p{0.3\linewidth}|} \hline

    \textbf{RS Stack} & \textbf{RS Type} \\ \hline \hline
    3 and 2: \textbf{Domain specific tools} and  \textbf{Scientific infrastructure} & 2 and 3: \textbf{Framework, Application} \\ \hline
    \multicolumn{2}{|l|}{\textbf{All from User story: Team OSS, Library}: Table \ref{tab:rs_recc06}} \\ \hline
    \multicolumn{2}{|l|}{\textbf{EOSC-SWTest-04}: Test-Driven Development (TDD)} \\ \hline
    \multicolumn{2}{|l|}{\textbf{EOSC-SWTest-10}: Scalability testing} \\ \hline

  \end{tabular}
  \caption{Recommended Quality Attributes, team OSS, frameworks and applications.}
  \label{tab:rs_recc07}
\end{table}

The Quality Attributes related to software release management are shown in Table \ref{tab:rs_rel}. In the case of an Open Source Software development team, software release and management is part of its responsibility. Thus, the additional recommended Quality Attributes include the existence of a helpdesk and/or bug tracking system, recommendation for software versioning, packaging and documentation. This reflects the RS lifecycle diagram \ref{fig:rslifecycle} boxes \textit{Maintenance}, \textit{Community Feedback} and \textit{Deployment}.

\begin{table}[h]
    \centering
    \scriptsize
    \begin{tabular}{|p{0.6\linewidth}|} \hline

        \textbf{EOSC-SWRelMan-09}: Semantic Versioning \\ \hline
        \textbf{EOSC-SWRelMan-13}: Register/publish artefact \\ \hline
        \textbf{EOSC-SWRelMan-17}: SCM tool \\ \hline
        \textbf{EOSC-SWRelMan-20}: Installability \\ \hline
        \textbf{EOSC-SWRelMan-25}: Provide checksums \\ \hline

    \end{tabular}
    \caption{Recommended Quality Attributes for SW release and management.}
    \label{tab:rs_rel}
\end{table}

\subsubsection{User story: Team Service}

This subsection refers to a medium to large team developing an Open Source scientific research service or platform that is used by many researchers, possibly from different fields. The list of Quality Attributes is significant with several types of tests including security, complete set of documentation, helpdesk for support and bug reporting. See RS lifecycle diagram \ref{fig:rslifecycle} boxes \textit{Maintenance}, \textit{Community Feedback} and \textit{Testing/CICD}.

\begin{table}[h]
  \centering
  \scriptsize
  \begin{tabular}{|p{0.55\linewidth}|p{0.3\linewidth}|} \hline

    \textbf{RS Stack} & \textbf{RS Type} \\ \hline  \hline
    3 and 2: \textbf{Domain specific tools} and \textbf{Scientific infrastructure} & 5: \textbf{Services and platforms} \\ \hline
    \multicolumn{2}{|l|}{\textbf{All from User story: Team Project}: Table \ref{tab:rs_recc05}} \\ \hline
    \multicolumn{2}{|l|}{\textbf{EOSC-TMet-04}: Defect rates} \\ \hline
    \multicolumn{2}{|l|}{\textbf{EOSC-SWRelMan-08}: Code review} \\ \hline
    \multicolumn{2}{|l|}{\textbf{EOSC-SWRelMan-26}: Documentation version controlled} \\ \hline
    \multicolumn{2}{|l|}{\textbf{EOSC-SWRelMan-27}: Documentation as code} \\ \hline
    \multicolumn{2}{|l|}{\textbf{EOSC-SWRelMan-28}: Documentation formats} \\ \hline
    \multicolumn{2}{|l|}{\textbf{EOSC-SWTest-04}: Test-Driven Development (TDD)} \\ \hline
    \multicolumn{2}{|l|}{\textbf{EOSC-SWTest-08}: Performance testing} \\ \hline
    \multicolumn{2}{|l|}{\textbf{EOSC-SWTest-09}: Stress testing} \\ \hline
    \multicolumn{2}{|l|}{\textbf{EOSC-SWTest-10}: Scalability testing} \\ \hline
    \multicolumn{2}{|l|}{\textbf{EOSC-SWTest-12}: Open Web Application Security Project (OWASP)} \\ \hline
    \multicolumn{2}{|l|}{\textbf{EOSC-SWTest-13}: Static Application Security Testing (SAST)} \\ \hline
    \multicolumn{2}{|l|}{\textbf{EOSC-SWTest-14}: Security code reviews} \\ \hline
    \multicolumn{2}{|l|}{\textbf{EOSC-SWTest-19}: API security assessment} \\ \hline
    \multicolumn{2}{|l|}{\textbf{EOSC-SWTest-21}: Dynamic Application Security Testing (DAST)} \\ \hline
    \multicolumn{2}{|l|}{\textbf{EOSC-SWTest-22}: Interactive Application Security Testing (IAST)} \\ \hline
    \multicolumn{2}{|l|}{\textbf{EOSC-SWTest-23}: Security penetration testing} \\ \hline
    \multicolumn{2}{|l|}{\textbf{EOSC-SWTest-24}: Security assessment} \\ \hline
    \multicolumn{2}{|l|}{\textbf{EOSC-SWTest-25}: Security as Code (SaC) Testing} \\ \hline

  \end{tabular}
  \caption{Recommended Quality Attributes, team service, services and platforms.}
  \label{tab:rs_recc08}
\end{table}

In the case of services and platforms in production, additional Quality Attributes are recommended. Table \ref{tab:rs_serviceprod} shows this list. They include the existence of a helpdesk and/or bug tracking system, monitoring, accounting, authentication and authorization mechanisms, documentation and policies to access and use the service or platform.

\begin{table}[h]
    \centering
    \scriptsize
    \begin{tabular}{|p{0.6\linewidth}|} \hline

        \textbf{EOSC-TMet-08}: \# Registered users \\ \hline
        \textbf{EOSC-TMet-09}: \# Active users \\ \hline
        \textbf{EOSC-TMet-10}: Amount computing resources \\ \hline
        \textbf{EOSC-TMet-11}: Amount storage resources \\ \hline
        \textbf{EOSC-SWRelMan-17}: SCM tool \\ \hline
        \textbf{EOSC-SWRelMan-20}: Installability \\ \hline
        \textbf{EOSC-SWRelMan-22}: Infrastructure as Code (IaC) validation \\ \hline
        \textbf{EOSC-SWTest-16}: Public endpoints and APIs encrypted \\ \hline
        \textbf{EOSC-SWTest-17}: Strong ciphers \\ \hline
        \textbf{EOSC-SWTest-18}: Authentication and Authorization \\ \hline
        \textbf{EOSC-SWTest-20}: Service compliance with data regulations (GDPR) \\ \hline
        \textbf{EOSC-SWTest-24}: Security assessment \\ \hline
        \textbf{EOSC-SrvOps-01}: Acceptable Usage Policy (AUP) \\ \hline
        \textbf{EOSC-SrvOps-02}: Access Policy and Terms of Use \\ \hline
        \textbf{EOSC-SrvOps-03}: Privacy policy \\ \hline
        \textbf{EOSC-SrvOps-04}: Operational Level Agreement (OLA) \\ \hline
        \textbf{EOSC-SrvOps-05}: Service Level Agreement (SLA) \\ \hline
        \textbf{EOSC-SrvOps-06}: Monitoring service public endpoints \\ \hline
        \textbf{EOSC-SrvOps-07}: Monitoring service public APIs \\ \hline
        \textbf{EOSC-SrvOps-09}: Monitoring security public endpoints and APIs \\ \hline

    \end{tabular}
    \caption{Recommended Quality Attributes for a service or platform in production.}
    \label{tab:rs_serviceprod}
\end{table}

%%%%%%%%%%%%%%%%%%%%%%%%%%%%%%%%%%%%%%%%%%%%%%%%%%%%%%%%%
% \eli{Portability. Depending on the design of the software.}

% \miguel{The problem of the containers (Docket et al.) and software development in online platforms. Advantage of containers: easier to releaseand to ensure reproducibility and to ensure the service. Disadvantages: containers freeze the environment, and thus they stop receiving bug fixing and security updates. Dependency with outdated software components.}

% \miguel{Stability of research SW. Does it ensure backwards compatibility between versions?}

% \cerlane{Traditionally “portability” can refer to the portability across hardware architecture, system and platform (compilers and libraries) by applying good software engineering practices. With containers, portability can also be achieved by restricting the supported system and platform in the container. A multi-architecture container can further improve portability.}

\newpage
\subsection{Example of tools, services and infrastructures}

\subsubsection{About version control and software forges}

Version control systems help track and manage changes to files and particularly to source code. A software forge is a web-based collaborative platform for developing and sharing codes.

The Quality Attributes related to version control are the following:

\begin{itemize}
  \item EOSC-SWRelMan-02: Version Control System (VCS)
  \item EOSC-SWRelMan-03: Source code hosting
  \item EOSC-SWRelMan-04: Working state version
\end{itemize}

Version control is an essential tool for software development. Software forges have become vital tools for all software developers. This type of infrastructure is often deployed at different levels; laboratory, project, institution, etc.. Commercial software forges (such as GitHub and GitLab), are significantly used in the academic context. A recent report of the Software and Sources Codes College of the Committee for Open Science illustrate the landscaping in France \cite{leberre:hal-04208924} shows this trend.

An important point is that most of the software forges integrate a large variety of tools (most of them listed below), that goes far beyond version control.

Software forges are a key platform to manage contributions to the code from many diverse people. They integrate the mechanism of Pull or Merge Requests, allowing any developer to contribute to the code and being peer reviewed, for the purpose of adding a new functionality, bug and security fixes, etc.

\subsubsection{About Continuous Integration/Continuous Delivery - CI/CD}
\label{sec_cicd}

Continuous Integration and Continuous Delivery (CI/CD), is one of the best practices in software development that implement the DevOps methodology. Changes are immediately and automatically tested and executed, and packages (or artefacts) are built and published in public repositories. 

CI/CD systems are the way to implement the verification of a set of Quality Attributes. The set of QAs to be verified, is implemented through a pipeline that is executed in one such CI/CD service. The CI/CD system is connected with the software forges, and the pipelines are triggered when there are changes in the source code.
Package building, testing, deployment, documentation linting, are some of the tasks that can be automated in CI/CD pipelines.

Examples of CI/CD systems are services such as; Jenkins (\url{https://www.jenkins.io/}), GitLab CI (\url{https://docs.gitlab.com/ee/ci/}) and Travis CI(\url{https://travis-ci.com/}).

The Software Quality as a Service (SQAaaS) platform (\url{https://sqaaas.eosc-synergy.eu}), is a service developed in EOSC-Synergy project\footnote{\url{https://www.eosc-synergy.eu/}.}, it allows Jenkins pipeline composition for automatic tests as well as Badge awarding according to predefined verification of Quality Attributes and allows to perform Verification \& Validation of the Software or Services.

\subsubsection{About intellectual property and licensing}

The quality attributes related to legal issues and licensing are:

\begin{itemize}
  \item EOSC-Qual-27: Intellectual Property (IP)
  \item EOSC-SWRelMan-01: Open Source
  \item EOSC-SWRelMan-10: Open-Source license
\end{itemize}

There is no real tool to verify the IP, where support can be provided by the institution. This support must be as large as possible including the help about Developer Certificate of Origin (DCO), or Contributor License Agreement (CLA).

Regarding Open Source and license, there are tools to verify that a given Open Source license file is present in the software forge, comparing it with a know database of Open Source licenses, one such tool is ``\textit{licensee}''\footnote{\url{https://github.com/licensee/licensee}}.

\subsubsection{About packaging}

Software packaging is the process to build a self-contained package or artefact of a software, in order for it to be easily installable.

The Quality Attributes regarding this topic are:

\begin{itemize}
  \item EOSC-SWRelMan-12: Packaging
  \item EOSC-SWRelMan-15: Code deployment
\end{itemize}

The number of packaging system is very large, thus we list just a few of such tools and services. We focus on mostly used tools in academic research and open
source context. This is {\bf not} an exhaustive list.

\begin{table}[h]
  \centering
  \scriptsize
  \begin{tabular}{|p{0.08\linewidth}|p{0.16\linewidth}|p{0.26\linewidth}|p{0.39\linewidth}|} \hline

    \textbf{Tools} & \textbf{Usage} & \textbf{URL} & \textbf{Comment} \\ \hline \hline
    \textbf{Guix} & Packaging system & \url{https://guix.gnu.org/en/} & Generate reproducible environment \\ \hline
    \textbf{Nix} & Packaging system & \url{https://nixos.org/} & Generate reproducible environment \\ \hline
    \textbf{CMake} & Build system & \url{https://cmake.org/} & Include many processes like compilation, packaging, testing  \\ \hline
    \textbf{Docker} & Container technology & \url{https://www.docker.com/} &  Certainly the most used \\ \hline
    \textbf{Singularity} & Container technology & \url{https://sylabs.io/} & Technology used in many computing centers \\ \hline
    \textbf{Kubernetes} & Container orchestration & \url{https://kubernetes.io/} & Mostly used in services deployment  \\ \hline
    \textbf{Openshift} & PaaS & \url{https://www.redhat.com/fr/technologies/cloud-computing/openshift} & Based on Docker and Kubernetes technologies  \\ \hline
    \textbf{RPM} & Package manager & \url{https://rpm.org/} & RedHat and derivatives packaging system \\ \hline
    \textbf{deb} & Package system & \url{https://wiki.debian.org/Packaging} & Debian/Ubuntu and derivatives packaging system \\ \hline

  \end{tabular}
  \caption{Tools and services for packaging and deployment}
  \label{tab:tools_pack01}
\end{table}

It should be noted that each of these tools, services and packaging systems are available or used in most research infrastructures.

\subsubsection{About documentation}

Documentation is of critical interest for software and covers many Quality Attributes: 

\begin{itemize}
  \item EOSC-SWRelMan-29: Documentation online
  \item EOSC-SWRelMan-30: Documentation updates
  \item EOSC-SWRelMan-32: Documentation production
  \item EOSC-SWRelMan-26: Documentation version controlled
  \item EOSC-SWRelMan-27: Documentation as code
  \item EOSC-SWRelMan-28: Documentation formats
\end{itemize}

The production of documentation is mostly manual but there are quite a few number of tools that help generate or produce it.

\begin{table}[h]
  \centering
  \scriptsize
  \begin{tabular}{|p{0.12\linewidth}|p{0.18\linewidth}|p{0.33\linewidth}|p{0.27\linewidth}|} \hline

    \textbf{Tools} & \textbf{Usage} & \textbf{URL} & \textbf{Comment} \\ \hline \hline
    \textbf{Doxygen} & Automatic generation & \url{https://www.doxygen.nl/} & Multi-languages \\ \hline
    \textbf{Sphinx} & Automatic generation & \url{https://www.sphinx-doc.org/en/master/} & For Python language \\ \hline
    \textbf{Javadoc} & Automatic generation & \url{https://www.oracle.com/java/technologies/javase/javadoc-tool.html} & For Java language  \\ \hline
    \textbf{MediaWiki} & Wiki engine & \url{https://www.mediawiki.org/wiki/MediaWiki} &  Available on GitLab platform \\ \hline
    \textbf{GitLab Pages} & Automatic generation & \url{https://docs.gitlab.com/ee/user/project/pages/} & Available in GitLab platform with CI  \\ \hline
    \textbf{Hugo} & Static web site generation & \url{https://gohugo.io/} & Based on Go language \\ \hline
    \textbf{Pelican} & Static web site generation & \url{https://getpelican.com/} & based on Python language \\ \hline

  \end{tabular}
  \caption{Tools and services for documentation}
  \label{tab:tools_pack02}
\end{table}

\subsubsection{About management bugs}

Bug tracking systems or more generally issue tracking systems are common components for software development. The relevant Quality Attributes are:

\begin{itemize}
  \item EOSC-TMet-02: \# Resolved bugs
  \item EOSC-TMet-03: \# Open bugs
\end{itemize}

Most software forges include a bug or issue tracking system. There are other tools such as Redmine (which is also a forge software like, \url{https://www.redmine.org/}), Trac (\url{http://trac.edgewall.org/}), GLPI (\url{https://www.glpi-project.org/}), Request Tracker (\url{https://bestpractical.com/request-tracker}) or Bugzilla (\url{https://www.bugzilla.org/}).

\subsubsection{About tests}

Automated Software testing is considered a best practice in software development. This can be implemented in CI/CD pipelines (Section~\ref{sec_cicd}) and covers the following Quality Attributes:

\begin{itemize}
  \item EOSC-SWTest-02: Unit tests
  \item EOSC-SWTest-03: Test doubles
  \item EOSC-SWTest-07: Functional testing
  \item EOSC-SWTest-05: API testing
  \item EOSC-SWTest-06: Integration testing
  \item EOSC-SWTest-08: Performance testing
  \item EOSC-SWTest-13: Static Application Security Testing (SAST)
  \item EOSC-SWTest-04: Test-Driven Development (TDD)
  \item EOSC-SWTest-10: Scalability testing
  \item EOSC-SWTest-09: Stress testing
\end{itemize}

There are a large number of tools and they cover several programming languages and frameworks, we list below a few of them:

\begin{itemize}
  \item For Python; tox (\url{https://tox.wiki}), pytest (\url{https://pytest.org/}).
  \item For Java; JUnit (\url{https://junit.org/}).
  \item For C++; CppUnit (\url{http://freedesktop.org/wiki/Software/cppunit}).
\end{itemize}

\subsubsection{About support}

The existence of a helpdesk is very important for users, operators, developers to create tickets about bugs, questions, new features, security issues, operation or usage of the software or service. The helpdesk contributes to the trust and adoption of the corresponding software or service. The Quality Attribute related to support is:

\begin{itemize}
  \item EOSC-SWRelMan-07: Support
\end{itemize}

There are many services that can be used for support and we list a few of them:

\begin{itemize}
  \item GitHub issues \url{https://docs.github.com/en/issues}: is an issue tracking system integrated within the GitHub repository
  \item GitLab issues \url{https://gitlab.com/gitlab-org/gitlab/-/issues}: is to GitHub but for GitLab.
  \item Jira from Atlassian \url{https://www.atlassian.com/software/jira}: is an issue and project tracking system.
  \item Request Tracker - RT \url{https://bestpractical.com/request-tracker}: Helpdesk.
  \item Zammad \url{https://zammad.com/en}: Ticketing system.
\end{itemize}

\subsubsection{About code analysis}

Static code analysis contribute to Software quality, the relevant Quality Attributes are the following:

\begin{itemize}
  \item EOSC-SCMet-02: \% of redundant code
  \item EOSC-SCMet-10: Number of comments
  \item EOSC-SWTest-01: Code style
  \item EOSC-SWTest-15: No world-writable files or directories
  \item EOSC-SWRelMan-08: Code review
\end{itemize}

There are many tools that can help to automate this analysis as part of CI/CD pipelines. These follow best practices and implement code conventions checking and well identified programming rules.

\begin{table}[h]
  \centering
  \scriptsize
  \begin{tabular}{|p{0.08\linewidth}|p{0.18\linewidth}|p{0.32\linewidth}|p{0.32\linewidth}|} \hline

    \textbf{Tools} & \textbf{Usage} & \textbf{URL} & \textbf{Comment} \\ \hline \hline
    \textbf{Pylint} & Static code analysis & \url{https://pylint.pycqa.org/} & For Python language. Others *lint tools exists for others languages : CPPlint for CPP, JSLint for JavaScript \\ \hline
    \textbf{Hadolint} & Dockerfile linter & \url{https://github.com/hadolint/hadolint} & To ensure best practice in Docker images \\ \hline
    \textbf{checkstyle} & Static code analysis & \url{https://checkstyle.sourceforge.net/} & For Java language  \\ \hline
    \textbf{pycodestyle} & Static code analysis & \url{https://pypi.org/project/pycodestyle/} & Check Python code against some of the style conventions in \href{http://www.python.org/dev/peps/pep-0008/}{PEP 8}  \\ \hline
    \textbf{flake8} & Static code analysis & \url{https://flake8.pycqa.org/en/latest/#} & Check Python code against some of the style conventions in \href{http://www.python.org/dev/peps/pep-0008/}{PEP 8} \\ \hline
    \textbf{SonarQube} & Automatic reviews with static analysis of code & \url{http://sonarqube.org/} & Supports many programming languages  \\ \hline
    \textbf{gcov} & Dynamic coverage analysis & \url{https://gcc.gnu.org/onlinedocs/gcc/Gcov.html} & Program must be compiled with specific options  \\ \hline
    \textbf{valgrind} & Memory error detection & \url{https://www.valgrind.org/} & Runs programs on a virtual processor  \\ \hline

  \end{tabular}
  \caption{Tools for static and dynamic code analysis}
  \label{tab:tools_pack03}
\end{table}

\subsubsection{About security}

Security is a critical topic especially for platforms and services, but also for other types of Software. Several Quality Attributes are related to security:

\begin{itemize}
  \item EOSC-TMet-04: Defect rates
  \item EOSC-SWTest-12: Open Web Application Security Project (OWASP)
  \item EOSC-SWTest-13: Static Application Security Testing (SAST)
  \item EOSC-SWTest-14: Security code reviews
  \item EOSC-SWTest-16: Public endpoints and APIs encrypted
  \item EOSC-SWTest-17: Strong ciphers
  \item EOSC-SWTest-18: Authentication and Authorization
  \item EOSC-SWTest-19: API security assessment
  \item EOSC-SWTest-20: Service compliance with data regulations (GDPR)
  \item EOSC-SWTest-21: Dynamic Application Security Testing (DAST)
  \item EOSC-SWTest-22: Interactive Application Security Testing (IAST)
  \item EOSC-SWTest-23: Security penetration testing
  \item EOSC-SWTest-24: Security assessment
  \item EOSC-SWTest-25: Security as Code (SaC) Testing
\end{itemize}

Some of the analysis tools include dynamic security analysis. Others, such as Splint (\url{http://www.splint.org/}, for C language) or  Bandit (\url{https://bandit.readthedocs.io/en/latest/} for Python language) can be used for static security code analysis.

\subsubsection{Transversal needs}

The tools and services listed in the previous sections are not enough to produce high quality software. Training is crucial at different level (students, researchers, and even software developers), in order to promote best practices and a good use of the different tools and services.

This can be done in different ways:

\begin{itemize}
  \item Introduction of specific courses in formal education.
  \item Lifelong training.
  \item On line training (such as MOOCs)
  \item Seminars, webinars, tutorials, hands-on courses.
\end{itemize}

It is also of crucial importance to develop proximity support for all scientific communities on these topics.
