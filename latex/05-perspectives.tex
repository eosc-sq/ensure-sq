%% massimo: I added the following paragraph to introduce what we mean (or at least what I understood) about "perspectives" and then introduce the plot of the discussion.  
Perspectives on research software quality include many different aspects that researchers and the scientific community at large consider relevant when evaluating the quality of software used or developed in a particular research field. These aspects are all equally essential to ensure the credibility and reliability of the research results themselves while meeting the expectations of software users. Clearly, the different perspectives on the quality of research software can branch significantly based on the individuals using it, the type of software, and the context in which it is used. 
In the following, we discuss perspectives according to two distict point of view: that of the developers of research software, and that of the users who may be either other researchers extending research into the same or different research field or mere software users. Finally, we discuss the aspects related to the services and resources that enable the development, the execution, and the fruition of the results produced by the software.  

\subsection{Developers}
% \miguel{Are the authors of the research SW obtaining an official acknowledgement of the value of their work? Does this have an impact on their career?}
% \miguel{Which are the licenses available? Which one should the developer use? Are the developers free to pick any SW license, or is it their employer who picks it?}
% \miguel{Does the developers of research SW have all the rights about the research SW they produce? Intellectual property (authoring), exploitation, re-use, re-licensing.}

Research software is developed by various individuals, e.g., researches and research engineers, as well as entities such as industries and academic laboratories, depending on the context and purpose of the software. 
The research software developer category includes, but is not limited to, the following: 
\begin{itemize}
\item \textbf{Academic Researchers}: Many researchers develop software tools or applications to support their research projects and ideas. This software is often tailored to specific research needs and may not be widely used outside of the research group. However, more and more often, a group of academic researchers, in some cases with heterogeneous expertise, work together within a research laboratory or project to build software tools and frameworks for their actual research and to foster the research of other individuals or research groups.

\item \textbf{Research Institutions}: Research software is sometimes developed by research institutions or research organizations (e.g., CNR in Italy or CNRS in France). These institutions often have dedicated software departments or laboratories that create software tools for various research projects.

\item \textbf{Nonprofit Organizations}: Some nonprofit organizations (e.g., Software Heritage) focus on the development of research software and tools that are made available to the research community free of charge or at a reduced cost, usually via donations.

\item \textbf{Research communities}: Open research communities are essential for advancing and promoting collaborations in the research and scientific communities. These communities comprise individuals, organizations, and institutions committed to open research principles, including open access to research publications, data sharing, open-source software, and open quality assurance assessment. 
\end{itemize}

As observed by J. Cohen et al.~\cite{TheFourPillarsOfRSE-2021} ``good research software can make the difference between valid, sustainable, reproducible research outputs and short-lived, potentially unreliable or erroneous outputs''.
Research software developers are devoted to producing high-quality software that facilitates or enables scientific research and technological advancements but also adheres to best practices in software development to ensure reliability, maintainability, and long-term viability. Hence, regardless of the developer category, the quality and reliability of research software are essential to ensure the validity and reproducibility of research results.

One crucial aspect for research software developers is acknowledging the value of research software and recognizing its authorship. This point may significantly impact the developers' progress in their careers. In many cases, the contributions of researchers who develop software are only acknowledged through citations in academic papers since research software is almost always directly or indirectly linked to one or more publications. 
Additionally, traditional academic evaluation systems often underestimate the production of research software, preferring to rely on other achievements. Recently, many institutions have recognized the importance of addressing this issue, fostering a growing sensibility in establishing practices to assess and acknowledge research software development. Such new sensitivity will help create a more inclusive and accurate assessment of modern research specifically for what concens research software development. 

Another important aspect is the developers' rights regarding the research software they produce. Do they have all the rights to the research software they produce or contribute to produce? This aspect has several implications regarding intellectual property, exploitation or results, re-use in other research fields or contexts, re-licensing all or part of the software, and so on. The answer to the question depends on several factors, including institutional policies, funding agreements, collaboration agreements, and the licensing choices made by the developers for their software. Developers working within academic institutions or other research organizations may be subject to institutional policies regarding intellectual property. In some cases, the institution may claim ownership of the software or have specific policies governing the rights of developers. 
Additionally, if the research software development is funded by external entities (such as government agencies, private organizations, or foundations), the funding agreement terms may stipulate the ownership and rights associated with the software. More and more often, some funding agencies require that software developed with their support be released as open-source.
Generally, developers retain copyright in their software unless institutional policies or employment contracts state otherwise. In that case, developers typically have the right to choose the license under which they release their software and thus determine how others can use, modify, and distribute it.

In light of all these aspects, it is of foremost importance for developers to be aware of and understand the legal and contractual aspects surrounding the software they create. 


\subsection{Users}
% \miguel{Documentation: how to install, and use the SW. Examples.}
% \miguel{How to cite research SW?}
% \miguel{Is the research SW easy to understand and modify? Is it possible to do so? SW licenses.}
% \miguel{Level of reproducibility of the research SW.}

Research software users are heterogeneous and can include individuals with different backgrounds and coming from disparate professions. However, the primary users of research software are usually researchers, research engineers, and scientists. 

The diversity of users highlights the interdisciplinary nature of research software, which often spans various scientific fields and application domains. Adequate documentation, hands-on tutorials, and the steady availability of user support are crucial for ensuring that research software meets the users' needs and contributes to advancing knowledge in different fields.  

%% maybe a subsection on reproduciability
From the user's perspective, reproducing the results of a study, potentially published in a scientific manuscript, involves having access to the software and data used to produce the results. 
Users should have access to the software's source code to inspect the algorithms used, verify their implementation, and understand how the software works in any aspect.
To this end, the availability of comprehensive documentation is essential. Users should have access to clear instructions on how to set up the software, install dependencies, compile the code, and reproduce specific experiments or analyses. The source-code documentation is certainly a plus to enhance transparency and reproducibility.

The access to the datasets used in the study is essential for reproducing results through the research software. Users should have access to the same data used by the authors to run the software. Moreover, knowledge of the computational environment, including the operating system, software versions, dependencies, and hardware specifications used to obtain the results, is essential to facilitate the user to reproduce the results. Containers technology and virtual environments are more and more often used, when possible, to help recreate the exact computational environment.

However, users of research software should also recognize that exact reproducibility may not always be achievable due to factors such as variability in data, slightly different hardware-software platforms, or even inherent randomness and complexities in the specific algorithms used. 
Collaborative and transparent practices within the research community and between the research software developers and the user community, may contribute to increasing the culture of reproducibility, ensuring the results produced using research software are easely verifiable.


\subsection{Service providers}

% \miguel{Examples of service providers for research SW: Galaxy project, Jupyter lab, IEEE's Code Ocean, supercomputing infrastructures (BSC, and others), CERN data analysis platform, ...}
% \miguel{Can the SW be put inside a container?}
% \miguel{Are the developers providing timely updates (bug fixes, security updates)}
% \cerlane{Question: Are we talking about service providers of the Infrastructure or research SW?}

In this section, a service provider refers to a service offered by an external organization to assist individual researchers and research entities in developing and releasing their research software. 
An external service provider may offer either physical resources or software tools and infrastructures.  
Several service providers meet the needs of researchers and developers engaged in research software development. Examples are GitHub, Zenodo, and AWS, to mention a few. Many software developers use such external services to enhance collaboration, streamline software development, maintain and disseminate the software they developed. 

However, although not every research software project needs to leverage services offered by service providers, the possibility of relying on such services without worrying about the maintainability of the service itself is undoubtedly an excellent value for research software developers.  
For example, researchers often rely on cloud services, such as AWS, Azure, and GCP, to access scalable computing infrastructures using a pay-as-you-need service model. These computing infrastructures are usually complex to manage and maintain, requiring specialized expertise and a consistent budget. Additionally, they provide flexibility in handling different workloads and optimizing resource utilization. 
Therefore, service providers enable the development and execution of the software and/or the fruition of the results produced by the software.

External services that represent an added value for research software can materialize in various forms. The following are some particularly significant services.
\begin{itemize}
\item \textbf{Code hosting and version control}. Examples are GitHub and GitLab, which offer version control, collaboration features (e.g., issue tracking), and code sharing. These platforms help to maintain a transparent development process. 
\item \textbf{Continuous Integration and Deployment}. Examples are Travis CI and Jenkins. These services automate the testing and deployment of software, ensuring code quality and streamlining the release process. 
\item \textbf{Containerization Services}. Examples are Docker Hub and Container Registries, which offer services that facilitate the packaging, distribution, and sharing of research software along with its dependencies, thus enhancing the reproducibility, portability, and scalability. 
\item \textbf{Cloud Computing Platforms}. Examples are Amazon Web Services (AWS), Microsoft Azure, Google Cloud Platform (GCP), which provide scalable infrastructure, enabling researchers to access computing resources on-demand. 
\item \textbf{Software and Data Repositories}. Examples are Zenodo, Dryad, Figshare, which offer archiving and sharing services for data and software, enhancing the accessibility of research artifacts. SoftwareHeritage is an example of archiving and preservation of the source code of software projects. It strongly emphasizes long-term preservation of source code, aiming to provide a reliable and persistent archive to ensure the availability of software code for future generations.
\item \textbf{Collaboration Platforms}. Examples are Slack and Microsoft Teams, which offer communication and collaboration tools that enhance team coordination, project management, and encourage real-time communication.
\item \textbf{High-Performance Computing Centers}. Several National and Regional HPC Centers provide researchers with access to powerful computing resources, enabling the execution of complex simulations and analyses that require substantial computational power.
\end{itemize}

There are also code review, training, and consultancy services for improving software development and maintenance. For example, Crucible facilitates code review processes to help maintain code quality, identify issues, and ensure that contributions meet established standards. Software Carpentry provides workshops, training events, and educational resources to help researchers improve their software and data skills. 

Given the vast amount of potentially valuable services for research software, one open issue is the fragmentation of such external services, i.e., the phenomenon where various software services, tools, or platforms are heterogeneous, often resulting in the parcellization of services with scarce interoperability and increasing the cost of accessing them. On the other hand, research software developers who heavily invest in one set of services may become locked into a particular platform or ecosystem, thus limiting their flexibility and introducing difficulties in migrating to alternative solutions. These phenomena can have important implications for research software developers. .

