\subsection{Quality Attributes related to SW release management}

Table \ref{tab:rs_rel} shows the list of recommended Quality Attributes for SW release and management. They include the existence of a helpdesk and/or bug tracking system, recommendation for software versioning, packaging and documentation. It's reflects the RS lifecycle diagram \ref{fig:rslifecycle} boxes \textit{Maintenance}, \textit{Community Feedback} and \textit{Deployment}.

\begin{center}
    \bottomcaption{Recommended Quality Attributes for software release anmanagement.}
    \label{tab:rs_rel}
    \small
    \begin{tabular}{|p{\linewidth}|}

        \textbf{EOSC-TMet-02}: \# Resolved bugs \\ \hline
        \textbf{EOSC-TMet-03}: \# Open bugs \\ \hline
        \textbf{EOSC-TMet-04}: Defect rates \\ \hline
        \textbf{EOSC-SWRelMan-07}: Support \\ \hline
        \textbf{EOSC-SWRelMan-09}: Semantic Versioning \\ \hline
        \textbf{EOSC-SWRelMan-12}: Packaging \\ \hline
        \textbf{EOSC-SWRelMan-13}: Register/publish artifact \\ \hline
        \textbf{EOSC-SWRelMan-17}: SCM tool \\ \hline
        \textbf{EOSC-SWRelMan-20}: Installability \\ \hline
        \textbf{EOSC-SWRelMan-25}: Provide checksums \\ \hline
        \textbf{EOSC-SWRelMan-29}: Documentation online \\ \hline
        \textbf{EOSC-SWRelMan-32}: Documentation production \\ \hline

    \end{tabular}
\end{center}

\subsection{Quality Attributes related to services and platforms in production}

Table \ref{tab:rs_serviceprod} shows the list of recommended Quality Attributes for services and platforms in production. They include the existence of a helpdesk and/or bug tracking system, monitoring, accounting, authentication and authorization mechanisms, documentation and policies to access and use the service or platform.

\begin{center}
    \tablehead{\hline}
    \tabletail{\hline}
    \bottomcaption{Recommended Quality Attributes for software release and management.}
    \label{tab:rs_serviceprod}
    \small
    \begin{tabular}{|p{\linewidth}|}

        \textbf{EOSC-TMet-02}: \# Resolved bugs \\ \hline
        \textbf{EOSC-TMet-03}: \# Open bugs \\ \hline
        \textbf{EOSC-TMet-04}: Defect rates \\ \hline
        \textbf{EOSC-TMet-08}: \# Registered users \\ \hline
        \textbf{EOSC-TMet-09}: \# Active users \\ \hline
        \textbf{EOSC-TMet-10}: Amount computing resources \\ \hline
        \textbf{EOSC-TMet-11}: Amount storage resources \\ \hline
        \textbf{EOSC-SWRelMan-07}: Support \\ \hline
        \textbf{EOSC-SWRelMan-17}: SCM tool \\ \hline
        \textbf{EOSC-SWRelMan-20}: Installability \\ \hline
        \textbf{EOSC-SWRelMan-22}: Infrastructure as Code (IaC) validation \\ \hline
        \textbf{EOSC-SWRelMan-29}: Documentation online \\ \hline
        \textbf{EOSC-SWRelMan-32}: Documentation production \\ \hline
        \textbf{EOSC-SWTest-16}: Public endpoints and APIs encrypted \\ \hline
        \textbf{EOSC-SWTest-17}: Strong ciphers \\ \hline
        \textbf{EOSC-SWTest-18}: Authentication and Authorization \\ \hline
        \textbf{EOSC-SWTest-20}: Service compliance with data regulations (GDPR) \\ \hline
        \textbf{EOSC-SWTest-24}: Security assessment \\ \hline
        \textbf{EOSC-SrvOps-01}: Acceptable Usage Policy (AUP) \\ \hline
        \textbf{EOSC-SrvOps-02}: Access Policy and Terms of Use \\ \hline
        \textbf{EOSC-SrvOps-03}: Privacy policy \\ \hline
        \textbf{EOSC-SrvOps-04}: Operational Level Agreement (OLA) \\ \hline
        \textbf{EOSC-SrvOps-05}: Service Level Agreement (SLA) \\ \hline
        \textbf{EOSC-SrvOps-06}: Monitoring service public endpoints \\ \hline
        \textbf{EOSC-SrvOps-07}: Monitoring service public APIs \\ \hline
        \textbf{EOSC-SrvOps-09}: Monitoring security public endpoints and APIs \\ \hline

    \end{tabular}
\end{center}

%%%%%%%%%%%%%%%%%%%%%%%%%%%%%%%%%%%%%%%%%%%%%%%%%%%%%%%%%
% \eli{Portability. Depending on the design of the software.}

% \miguel{The problem of the containers (Docket et al.) and software development in online platforms. Advantage of containers: easier to releaseand to ensure reproducibility and to ensure the service. Disadvantages: containers freeze the environment, and thus they stop receiving bug fixing and security updates. Dependency with outdated software components.}

% \miguel{Stability of research SW. Does it ensure backwards compatibility between versions?}

% \cerlane{Traditionally “portability” can refer to the portability across hardware architecture, system and platform (compilers and libraries) by applying good software engineering practices. With containers, portability can also be achieved by restricting the supported system and platform in the container. A multi-architecture container can further improve portability.}

% DONE in previous sections \cerlane{Reproducibility is easier to achieve on the same system, software and hardware. However, it doesn't always imply a complete numerical reproducibility. To achieve numerical reproducibility, one would have to even avoid the use of certain optimisation compiler options, optimised Maths  libraries and apply very good software engineering techniques when applying algorithms, particularly with MPI.}
% DONE \miguel{Scalability. Depending on the volume of the data or the computational needs, it's a matter of system design, but eventually of the quality of the platform/software.}
% DONE \miguel{Quality of documentation: is the documentation of the research SW always synchronized with its release?}
% DONE \dg{[Jael, Daniel] Make sure this work aligns with SG1, which takes into account the software research cycle.}
% DONE \miguel{Does the SW follow some known methodology which includes verifications in order to ensure it works well? At which level? For example, if it uses continuous integration, does it include any step to ensure that the results are scientifically valid, or just unit tests?}
% DONE \miguel{Does the SW perform self tests? Say, unit tests.}
% DONE \cerlane{Do we mean 'Maintenance' or 'Maintainability'?}
% DONE \mdavid{We will take the SW lifecycle diagram from SG1 and connect it here, or some of the previous sections.}
% DONE partly in previous sections the link between RS and publications \miguel{Is the research SW associated with any scientific papers?
% For example, k-means clustering in scikit-learn \url{https://scikit-learn.org/stable/modules/clustering.html\#k-means}) is associated with:
% k-means++: The advantages of careful seeding” Arthur, David, and Sergei Vassilvitskii, Proceedings of the eighteenth annual ACM-SIAM symposium
% on Discrete algorithms, Society for Industrial and Applied Mathematics (2007)
% The source code provides the author names, but no reference paper:
% \url{https://github.com/scikit-learn/scikit-learn/blob/main/sklearn/cluster/_kmeans.py}
% However, is it associated with any other paper which provides a detailed procedure and a pseudocode? Without this, one can't have a good level of reproducibility.}
