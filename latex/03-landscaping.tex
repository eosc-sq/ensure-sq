\subsection{Definition and references of Research Software}
\label{subsec:defrs}

Research Software is not only software. To be able to clearly identify quality attributes for Research Software we rely on its definition given in page \ref{def_rs}.

The most important characteristic is the link with scientific publications. The purpose of Research Software is that it supports the scientific results, and from that it is expected to reproduce the published results. Regular software does not necessarily have this expectation.

The goal of this section is to cross-reference the elements of software quality from the previous section with the \textit{sub-group 1} \cite{sg1tf2023} (``Software Lifecycle'', c.f. Section \ref{sec:introduction}), typology of Research Software development, in order to identify the tools and infrastructures needed to satisfy minimum quality criteria, taking into account the specificities of Research Software.  

As a reminder, \textit{sub-group 1} has defined different categories of software development related to the context in which they are performed:

\begin{enumerate}
    \item \textbf{Individual development} - Individual creating Research Software for his/her own use (e.g. a PhD student): Based on a research question, software is created by a single person with the specific aim of answering the research question and producing research output (paper, dataset, etc).
    \item \textbf{Team Project} - A research team creating an application or workflow for the use within the team: Research Software is created by a team to answer a series of research questions (often as part of a larger research project).
    \item \textbf{Team OSS} - A team / community developing (possibly broadly applicable) open source Research Software: Software is created by a team (possibly distributed over multiple organisations) to answer a broad range of research questions.
    \item \textbf{Team Service} - A team or community creating a research service: A service platform is a set of software components which are used to provide services for a large number of users, most of whom make use of these offerings via the Internet (e.g. cloud services).
\end{enumerate}

Although there are other references, for instance \footnote{\url{https://docs.google.com/presentation/d/1uwxSwd8chbG7bVn5lPvNNhv5f_JeeelA2mN9NXhlOhA}} define similar or equivalent categories as enumerated above while \footnote{\url{https://zenodo.org/record/7589725}} rely on the same definition of RS as we did in page \ref{def_rs}.

\subsection{Research Software stack}

A typical scientific software stack~\cite{hinsen2019} has the composition shown in Table~\ref{tab:rs_stacks}.

\begin{table}[h]
    \centering
    \scriptsize
    \begin{tabular}[t]{|p{0.15\linewidth}|p{0.5\linewidth}|p{0.25\linewidth}|} \hline

    \textbf{Stack} & \textbf{Stack definition} & \textbf{Types} \\ \hline \hline
    4 - Project specific code &
    Software written by scientists for a specific research project. It can take various forms including scripts, notebooks, and workflows, but also special-purpose libraries and utilities. &
    Library; Analysis script and workflows; Services and platforms \\ \hline

    3 - Domain specific tools &
    Tools and libraries that implement models and methods which are developed and used by specific communities. GROMACS, MMTK, Amber. &
    Library; Application; Services and platforms \\ \hline

    2 - Scientific infrastructure &
    Infrastructure created specifically for scientific computing, but not for any particular application domain: mathematical libraries such as BLAS, LAPACK, or SciPy, scientific data management tools such as HDF5. &
    Library; Framework; Services and platforms\\ \hline

    \it{1 - Non-scientific infrastructure} &
    \it{Compilers and interpreters, libraries for data management; gcc, python...} &
    -- \\ \hline

    \end{tabular}
    \caption{RS stacks, stacks definitions and RS types. Stack level 1 is not considered in this report.}
    \label{tab:rs_stacks}
\end{table}

Note that Stack level 1 is not considered in this report. The Research Software types are summarised next including some examples:

\begin{enumerate}
    \item \textbf{Library}: SciPy, TensorFlow, FFTW, BLAS, LAPACK.
    \item \textbf{Framework}: JupyterLab, RStudio.
    \item \textbf{Application} (such as Monte-Carlo simulation Software): Gromacs, Amber, GEANT4, COMSOL.
    \item \textbf{Analysis script and workflows}
    \item \textbf{Services and platforms}: Galaxy server, Scipion, SAPS, O3AS.
\end{enumerate}


\subsection{Expectation for Research Software stack and types}

To be able to identify quality attributes for Research Software, we need to understand clearly what is expected from the different types identified in the previous sections.

\subsubsection{By Research Software type}

The different expectations described below add up to each other, when the context of development grow in complexity.

\textbf{Individual researcher creating software for personal use}

The author must be able to manage the different versions of the code. He/She often has to come back to the software after a long time (to answer to a reviewer for a publication, to address a new scientific question or confirm previous results).

He/She perhaps needs to execute the software on different computers and different operating systems or versions (for example the use of a supercomputer).

He/She has to share the code when he/she submits his article.

\textbf{A research team creating an application or workflow for use within the team}

The team must be able to manage the development with several people. The way to program should be homogeneous, and everybody must have a clear vision of who does what.

A new arrival in the team must be able to easily understand the structure of the software and the way to develop its code.

\textbf{A team / community developing Open Source Software}

For a community Research Software, it is important to facilitate different types of contribution and to manage correctly the rights attribution.

It's important to take into account Validation \& Verification of the code. The code has to be easy to maintain and to upgrade.

A community software must be as easy as possible to install, use and report bugs, problems or suggestions/new features. Strong software support must be in place to help the users.

\textbf{A team or community creating a service, a platform or an infrastructure}

The main expectation is availability, reliability and security of the service or platform for users and operators.

\textbf{A team / community developing software in an industrial context}

In the case of industrial context, expectations can vary depending on the contract between academic researchers and the company.

\subsubsection{Layer by layer in the software stack}

Regarding this categorisation, the user's point of view supersedes the developer's point of view.

\textbf{A scientific infrastructure}

A scientific infrastructure, as described in \cite{hinsen2019}, is by definition a software used by other software. That is, it's not a standalone code. Therefore it must be easy to install and to interface/integrate with other software.

Users expect this type of software to be robust and to be sufficiently sustainable since it's often the foundation of their own code.

\textbf{A domain specific tool}

A domain specific tool is similar to that of a community software. It must be easy to install on different operating systems and easy to use. It should have good retro-compatibility. It should be easy to cite. It's important for users to be able to know precisely what the software does without having to read the source. 

\textbf{A project specific code}

A project specific code has similar characteristics and requirements to that of a single person or team developing the code. Reproducibility plays an important role. It must be easily citable and archived for sustainability reasons. It often depends on the scientific infrastructures.
