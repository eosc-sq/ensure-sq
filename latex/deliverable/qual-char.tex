\begin{enumerate}
    \item \textbf{Functional suitability}; degree to which a product or system provides functions that meet stated and implied needs when used under specified conditions~\cite{iso_25010_2011_2017}.

    \item \textbf{Availability}; degree to which a system, product, or component is operational and accessible when required for use. We adapted this definition from~\cite{iso_iec_24765_2017}.

    \item \textbf{Reliability}; degree to which a system, product, or component provides specific functions under concrete conditions for a specified period of time. This definition is adapted from~\cite{iso_iec_24765_2017}. Lack of reliability can sometimes be caused by flaws in the requirements, the design, and the implementation, as well as because of contextual changes.  

    \item \textbf{Time behaviour}; degree to which the response, processing times, and throughput rates of a product or system meet the requirements when performing its functions~\cite{iso_25010_2011_2017}.

    \item \textbf{Performance}; related to how well the system works taking into account the amount of resources used under stated conditions~\cite{iso_25010_2011_2017}. The \textit{resources} might include other software products, the software and hardware configuration of the system, and any needed supplies (as for example print paper or storage media).

    \item \textbf{Ease of use (or \textit{usability})}; degree to which a product or system can be used by particular users to achieve their own goals with effectiveness, efficiency, and satisfaction in a given context of use. Usability can either be specified or measured as a product quality characteristic in terms of its sub-characteristics, or specified or measured directly by a subset of System/Software Product Quality~\cite{iso_25010_2011_2017}.

    \item \textbf{Fault tolerance}; degree to which a system, product, or component operates as intended despite the presence of hardware or software faults. We adapted this definition from ~\cite{iso_iec_24765_2017}.

    \item \textbf{Security}; degree to which a product or system protects the information and data it manages (say, stores or transmits) in a way that access is only given to persons or systems with the appropriate level of authorisation they were granted~\cite{iso_25010_2011_2017}. Related to Security we can also find:
    \begin{itemize}
        \item \textbf{Survivability}; degree to which a product or system continues to fulfil its mission by providing essential services in a timely manner despite the presence of attacks, covered by \textbf{Recoverability}.
        \item \textbf{Immunity}; degree to which a product or system is resistant to certain attacks, covered by \textbf{Integrity}.
    \end{itemize}
    
    \item \textbf{Confidentiality}; degree to which a product or system ensures that data are accessible only to those who have been authorised access~\cite{iso_25010_2011_2017}. This characteristic is a Security characteristic specialised on the privacy of the data.

    \item \textbf{Maintainability}; degree of effectiveness and efficiency with which a product or system can be modified by the intended maintainers~\cite{iso_25010_2011_2017}. Modifications can include corrections, improvements, or adaptation of the software to changes in the environment, and in the requirements and functional specifications. Modifications include both those carried out by specialised support staff, as well as those carried out by business or operational staff or end users. Maintainability can be interpreted as either an inherent capability of the product or system to facilitate maintenance activities, or the quality-in-use experienced by the maintainers for the goal of maintaining the product or system. It includes installation of updates and upgrades. 

    \item \textbf{Recoverability}; degree to which, in the event of an interruption or a failure, a product or system can recover the directly affected data and re-establish the desired state of the system~\cite{iso_25010_2011_2017}. Following a failure, a computer system will typically be down for some time, which is determined by its \textit{recoverability}.

    \item \textbf{Operability} and \textbf{Manageability}; degree to which a product or system has attributes that make it easy to operate and control~\cite{iso_25010_2011_2017}. \textit{Operability} corresponds to \textit{Controllability}, that is, the operator's error tolerance and the conformity with users' expectations.

    \item \textbf{Resource utilisation}; degree to which the amount and types of resources consumed by a product or system when performing its functions, meet the requirements~\cite{iso_25010_2011_2017}. Human resources are included in this category.

    \item \textbf{Safety}; degree to which a product or system mitigates the potential personal risk to humans or to system components, in the intended contexts of use~\cite{iso_25010_2011_2017}.

    \item \textbf{Interoperability}; degree to which two or more systems, products, or components can share data, encoded in agreed formats. This definition was adapted from ~\cite{iso_iec_24765_2017}.

    \item \textbf{Attractiveness}; degree to which a user interface enables pleasing and satisfying interaction with the user~\cite{iso_25010_2011_2017}. For example, one can include here properties of the product or system such as the use of colour and the nature of the graphical design. This characteristic is also known as \textit{interface aesthetics}.

    \item \textbf{Compatibility}; degree to which a product, system, or component can exchange information with other products, systems or components, and performs its designed functions, sharing the same hardware or software environment~\cite{iso_25010_2011_2017}. This definition is very close to the one of \textit{Interoperability}. Here we focus on the ability of the software to perform its functions in a suitable environment, whereas in Interoperability we refer to the format of the data which is exchanged.

    \item \textbf{Installability}; degree of effectiveness and efficiency a product or system can be successfully added or removed in a specified environment~\cite{iso_25010_2011_2017}.

    \item \textbf{Accessibility}; degree to which a product or system can be used by people with the widest range of diversity and capabilities to achieve a specified goal in a concrete context of use~\cite{iso_25010_2011_2017}. The range of capabilities includes diversity associated with age or any functional diversity. Accessibility can be specified or measured either as the extent to which a product or system can be used by users with diverse capabilities to achieve specified goals with effectiveness, efficiency, freedom from risk and satisfaction in a specific context of use, or by the presence of properties supporting specifically accessibility in the product.

    \item \textbf{Portability / Adaptability}; degree of effectiveness and efficiency with which a system, product, or component can be transferred from one hardware, software or other operational or usage environment to a different one~\cite{iso_25010_2011_2017}. Portability can be interpreted as either an inherent capability of the product or system to facilitate porting activities, or the quality-in-use experienced for the goal of porting the product or system.

    \item \textbf{Modifiability}; degree to which a product or system can be effectively and efficiently modified without introducing defects or degrading the product~\cite{iso_25010_2011_2017}.

    \item \textbf{Reusability}; degree to which a software artefacts (say, code, executable files, or any assets) can be exploited in other systems, or utilised to build other artefacts~\cite{iso_25010_2011_2017}.

    \item \textbf{Scalability}; this characteristic can be seen from the point of view of the system, or the applications. In the first case
    it's the measure of a system's ability to increase or decrease in performance and cost in response to changes in application and system processing demands. Examples would include how well a hardware system performs when the number of users is increased, how well a database withstands growing numbers of queries, or how well an operating system performs on different classes of hardware. Enterprises that are growing rapidly should pay special attention to scalability when evaluating hardware and software~\cite{gartner_2021}. When referring to algorithms, protocols, or applications it can be defined as being able to efficiently handle a growing demand of work or need of more performance by means of adding more resources to the system on which the software is running. Resources can be added both to single nodes (vertical scalability) and to the system as a whole (horizontal scalability)~\cite{bondi_2000}.

    \item \textbf{Supportability}; is it defined as the ability of the system to provide helpful information  to identify and resolve issues in case of malfunction~\cite{microsoft_2010}. The existence of a helpdesk,  issue tracking, bug reporting, or related services also contribute to supportability~\cite{orviz_fernandez_eosc-synergy_2020}.

    \item \textbf{Testability}; degree of effectiveness and efficiency with which test criteria can be established for a system, product, or component. Then the tests might be run to determine whether the criteria have been met~\cite{iso_25010_2011_2017}.
\end{enumerate}
