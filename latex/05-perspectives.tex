%% massimo: I added the following paragraph to introduce what we mean (or at least what I understood) about "perspectives" and then introduce the plot of the discussion.  
Perspectives on research software quality include many different aspects that researchers and the scientific community at large consider relevant when evaluating the quality of software used or developed in a particular research field. These aspects are all equally essential to ensure the credibility and reliability of the research results themselves while meeting the expectations of software users. Clearly, the different perspectives on the quality of research software can branch significantly based on the individuals using it, the type of software, and the context in which it is used. 
In the following, we discuss perspectives according to three different points of view: that of the developers of research software, that of the users who may be either other researchers extending research into the same or different research field or mere users of the software, and finally the point of view of the providers of the services and resources that enable the execution of the software and/or the fruition of the results produced by the software.  

\subsection{Developers}

Research software is developed by various individuals, e.g., researches and research engineers, as well as entities such as industries and academic laboratories, depending on the context and purpose of the software. 
The research software developer category includes, but is not limited to, the following: 
\begin{itemize}
\item \textbf{Academic Researchers}: Many researchers develop software tools or applications to support their research projects and ideas. This software is often tailored to specific research needs and may not be widely used outside of the research group. However, more and more often, a group of academic researchers, in some cases with heterogeneous expertise, work together within a research laboratory or project to build software tools and frameworks for their actual research and to foster the research of other individuals or research groups.

\item \textbf{Research Institutions}: Research software is sometimes developed by research institutions or research organizations (e.g., CNR in Italy or CNRS in France). These institutions often have dedicated software departments or laboratories that create software tools for various research projects.

\item \textbf{Nonprofit Organizations}: Some nonprofit organizations (e.g., Software Heritage) focus on the development of research software and tools that are made available to the research community free of charge or at a reduced cost, usually via donations.

\item \textbf{Research communities}: Open research communities are essential for advancing and promoting collaborations in the research and scientific communities. These communities comprise individuals, organizations, and institutions committed to open research principles, including open access to research publications, data sharing, open-source software, and open quality assurance assessment. 
\end{itemize}


Research software developers are devoted to producing high-quality software that facilitates or enables scientific research and technological advancements but also adheres to best practices in software development to ensure reliability, maintainability, and long-term viability. Hence, regardless of the developer category, the quality and reliability of research software are essential to ensure the validity and reproducibility of research results.

%%%%%%%%%%%%%%%%%%%%%%%%%%%%%
% \miguel{Are the authors of the research SW obtaining an official acknowledgement of the value of their work? Does this have an impact on their career?}
% \miguel{Which are the licenses available? Which one should the developer use? Are the developers free to pick any SW license, or is it their employer who picks it?}
% \miguel{Does the developers of research SW have all the rights about the research SW they produce? Intellectual property (authoring), exploitation, re-use, re-licensing.}

\subsection{Users}

% \miguel{Documentation: how to install, and use the SW. Examples.}
% \miguel{How to cite research SW?}
% \miguel{Is the research SW easy to understand and modify? Is it possible to do so? SW licenses.}
% \miguel{Level of reproducibility of the research SW.}

\subsection{Resource/Service providers}

% \miguel{Examples of service providers for research SW: Galaxy project, Jupyter lab, IEEE's Code Ocean, supercomputing infrastructures (BSC, and others), CERN data analysis platform, ...}
% \miguel{Can the SW be put inside a container?}
% \miguel{Are the developers providing timely updates (bug fixes, security updates)}
% \cerlane{Question: Are we talking about service providers of the Infrastructure or research SW?}
