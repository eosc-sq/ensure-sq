\subsection{Metadata schema for SW}

(Codemeta, other(s)...)

\subsection{The case for FAIR for Research Software (FAIR4RS)}

F: Software, and its associated metadata, is easy for both humans and machines to find.
F1. Software is assigned a globally unique and persistent identifier.
F1.1. Components of the software representing levels of granularity are assigned distinct identifiers.
F1.2. Different versions of the software are assigned distinct identifiers.
F2. Software is described with rich metadata.
F3. Metadata clearly and explicitly include the identifier of the software they describe.
F4. Metadata are FAIR, searchable and indexable.

A: Software, and its metadata, is retrievable via standardized protocols.
A1. Software is retrievable by its identifier using a standardized communications protocol.
A1.1. The protocol is open, free, and universally implementable.
A1.2. The protocol allows for an authentication and authorization procedure, where necessary.
A2. Metadata are accessible, even when the software is no longer available.

I: Software interoperates with other software by exchanging data and/or metadata, and/or
through interaction via application programming interfaces (APIs), described through
standards.
I1. Software reads, writes and exchanges data in a way that meets domain-relevant community standards.
I2. Software includes qualified references to other objects.

R: Software is both usable (can be executed) and reusable (can be understood, modified, built
upon, or incorporated into other software).
R1. Software is described with a plurality of accurate and relevant attributes.
R1.1. Software is given a clear and accessible license.
R1.2. Software is associated with detailed provenance.
R2. Software includes qualified references to other software.
R3. Software meets domain-relevant community standards.

Table 1: The FAIR Principles for Research Software


\subsection{Quality Attributes and the FAIR4RS principles}


% \textcolor{red}{DG: To be developed further once the quality metrics and criteria is clear. The scope of this section is not
% defining metadata for software, but stating how capturing these metadata may help addressing some of the quality metrics.
% For example, capturing schema:keywords may help having a complete description (not the greatest example, I know).}
