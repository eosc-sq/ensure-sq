\subsection{Definition and references of Research Software}

\dg{[summary discussion] The expectation of research software is that it can support the findings in a paper. In that sense, RS is "expected" to
reproduce those findings, while regular software would not necessarily have this expectation (as long as it works).}

\mdavid{\textbf{Review this subsection}}

The goal of this section is to cross-reference the elements of software quality from the previous section with the subgroup 1 \cite{sg1tf2023} ("Software Lifecycle", see section \ref{sec:introduction}), typology of research software development, in order to identify the tools and infrastructures needed to satisfy minimum quality criteria. 

As a reminder, subgroup 1 has defined different categories of software development related to the context in which they are performed. Others have different categories as well, for instance \url{https://docs.google.com/presentation/d/1uwxSwd8chbG7bVn5lPvNNhv5f_JeeelA2mN9NXhlOhA/edit#slide=id.g183ddf12e8b_0_122} and \url{https://zenodo.org/record/7589725#.ZGc5E3bP2Uk}

\subsection{Research Software stack and types}

A typical scientific software stack~\cite{hinsen2019} has the following composition:

\begin{center}
    \tablehead{
        \hline
        \textbf{Stack} & \textbf{Stack definition} & \textbf{Types} \\
        \hline
    }
    \tabletail{\hline}
    \tablelasttail{\hline}
    
    \bottomcaption{RS stacks, stacks definitions and RS types. Stack level 1 is not considered in this report.}
    \label{tab:rs_stacks}

    \small
    \begin{supertabular}{|p{0.15\linewidth}|p{0.5\linewidth}|p{0.3\linewidth}|}

    \hline
    4 - Project specific code &
    Software written by scientists for a specific research project. It can take various forms including scripts, notebooks, and workflows, but also special-purpose libraries and utilities. &
    Library; Analysis script and workflows; Services and platforms \\ \hline

    3 - Domain specific tools &
    Tools and libraries that implement models and methods which are developed and used by specific communities. Gromacs, MMTK, Amber. &
    Library; Application; Services and platforms
    \\ \hline

    2 - Scientific infrastructure &
    Infrastructure created specifically for scientific computing, but not for any particular application domain: mathematical libraries such as BLAS, LAPACK, or SciPy, scientific data management tools such as HDF5. &
    Library; Framework; Services and platforms\\ \hline

    \it{1 - Non-scientific infrastructure} &
    \it{Compilers and interpreters, libraries for data management; gcc, python...} &
    -- \\ \hline

    \end{supertabular}
\end{center}

Note that Stack level 1 is not considered in this report. The Research Software types are summarised next including some examples:

\begin{enumerate}
\item Library: SciPy, TensorFlow, FFTW, BLAS, LAPACK.
\item Framework: JupyterLab, RStudio.
\item Application (such as Monte-Carlo simulation Software): Gromacs, Amber, GEANT4, COMSOL.
\item Analysis script and workflows
\item Services and platforms: Galaxy server, Scipion, SAPS, O3AS.
\end{enumerate}

\subsection{User stories}
\label{subsec:user_stories}

This document takes the \textit{User Stories} detailed in subgroup 1~\cite{sg1tf2023} and summarised next:

\begin{enumerate}
    \item \textbf{Individual} - Individual creating research software for own use (e.g. a PhD student): Based on a research question, software is created by a single person with the specific aim of answering the research question and producing research output (paper, dataset, etc).
    \item \textbf{Team Project} - A research team creating an application or workflow for use within the team: Research software is created by a team to answer a series of research questions (often as part of a larger research project).
    \item \textbf{Team OSS} - A team / community developing (possibly broadly applicable) open source research software: Software is created by a team (possibly distributed over multiple organizations) to answer a broad range of research questions.
    \item \textbf{Team Service} - A team or community creating a research service: A service platform is a set of software components which is used to provide services for a large number of users, most of whom make use of these offerings via the Internet (e.g. cloud services).
\end{enumerate}




\subsection{Users stories previous text}

\mdavid{This was previously in section 4, and needs to be reviewed systematically to associate with the QAttrs of Appendix}

\textbf{Individual researcher creating software for personal use} 

The expected quality criteria are minimal but essential to ensure
reuse and continuity of developments (usually by another doctoral
student and/or by the research team), as well as the possibility of
reproducing the results of associated publications. 

{\bf Practices:}
\begin{itemize}
\item Version Control System
\item Basic documentation Documentation de base (including comments in the code)
\item License 
\item archiving and citation
\item Publication on the code
\end{itemize}

{\bf Tools and infrastructure required:}
\begin{itemize}
\item Software forge (with public access) - For example gitlab instance
\item Automatic documentation generator - For example doxygen
\item Archiving infrastructure (with perennial id) - For example Software Heritage
\item Cataloguing infrastructure - For example HAL en France
\item Journal for the publication on codes - For example JOSS
\end{itemize}

% \violaine{Add needed skills -> need training at PhD, researcher and engineer level \\
% Add needed support -> help to identify the adapted infrastructures and how to use them}

\textbf{A research team creating an application or workflow for use within the team}

The quality criteria are more demanding because it is a question here
of ensuring continuity in the development carried out by several
people, over more or less long periods (researchers, engineers,
doctoral students, trainees...).

In addition to the above elements, more advanced practices must be
integrated.

{\bf Additional practices:}
\begin{itemize}
\item Release management
\item Testing (unit, functional, regression)
\item Continuous integration
\item Programming rules
\item Bug tracking
\item License management
\end{itemize}

{\bf Additional tools and infrastructure required:}
\begin{itemize}
\item Continuous integration module - For example Jenkins or gitlab CI
\item Issue tracking system - For example Jira, Redmine ...
\item Static web site - For example gitlab pages 
\item License compatibility tool - For example flict
\end{itemize}

% \violaine{Add needed skills and support as about license, CI ...}

\textbf{A team / community developing open source software}

The software is developed within the framework of a community,
generally distributed, whose actors have different roles. In general,
there is a software governance that organizes the developments. The
aim is to ensure optimal quality so that the whole community can use
the software with confidence and also make contributions.

{\bf Additional practices:}
\begin{itemize}
\item Easy installation
\item User and developer documentation
\item Contribution tracking / Copyright management
\item Regular releases
\item Code review
\item Code coveraging
\item Code quality check
\item Possibility of online tutorials / avaibility of examples
\item User support
\item Ensure visibility
\item Ensure good development practices and compliance with standards
\end{itemize}

{\bf Additional tools and infrastructure required:}
\begin{itemize}
\item Software forge allowing contributions to the largest number
\item Test server - For example Cdash ...
\item Packaging system (such as guix, nix, spack ...) / VM with these systems installed
\item Container / Kubernetes
\item Docker registry
\item Code coverage tool
\item Static code analysis tool
\item Notebook server
\item Forum / chat tool
\item Code quality verification tool - For example SonarQube
\item DCO (Developer Certificate of Origin) or CLA (Contributor License Agreement) / Copyright
\item Synchronization between forges - For example Fediverse, Forge Friends ...
\item Tool for detecting bad development practices / non compliance with
standards
\end{itemize}

Software examples - by far not exhaustive - in different communities
are: Astropy1 (astronomy);  ESMValTool2 (Earth sciences); VASP3,
FHI-aims4 (material research); VIP5 (medical sciences); Gromacs6
(molecular dynamics); Neuron7, Nest8, Genesis (Neuroinformatics); root
(particle physics); many more examples can be found at, e.g., the
Research Software Directory

% \violaine{Link with the computing centers for support of packaging system or container}

\textbf{A team or community creating a service, a platform or an infrastructure}

The case is quite similar to the previous one but includes additional
constraints linked to the provision of a service. The necessary means
in operational terms are more important. 

{\bf Additional practices:}
\begin{itemize}
\item Viable business model
\item Security of the service/platform
\end{itemize}

{\bf Additional tools and infrastructure required:}
\begin{itemize}
\item Storage
\item VM / Container
\end{itemize}

\textbf{A team / community developing software in an industrial context}

The expected quality criteria are very similar to the case of
community developments. Depending on the nature of the collaboration
with the industry, the security aspects may become essential. 

{\bf Additional practices:}
\begin{itemize}
\item Code certification
\item Dependency security
\end{itemize}

{\bf Additional tools and infrastructure required:}
\begin{itemize}
\item Certification Infrastructure
\item Vulnerability Detection Tool
\end{itemize}

