\subsection{Quality Attributes recommendations}


\subsubsection{Common recommendations}

Table~\ref{tab:rs_recc00} shows the list of Quality Attributes that are common to all types of Software, and to all user stories described in subsection~\ref{subsec:defrs}. They are deemed important and not too difficult to implement even by an individual research or software developer.

These attributes are about documentation, open source code including a license, visible description about intellectual property that is important to establish ownership and scientific recognition. Also avoid redundant code and about security; no world writable files and directories.

\begin{center}
    \tablehead{\hline \textbf{RS Stack} & \textbf{RS Type} \\ \hline}
    \tabletail{\hline}
    \bottomcaption{Recommended common Quality Attributes.}
    \label{tab:rs_recc00}
    \small
    \begin{supertabular}{|p{0.65\linewidth}|p{0.35\linewidth}|} \hline
    \textbf{All} &
    \textbf{All} \\ \hline \hline

    \multicolumn{2}{|l|}{\textbf{EOSC-SCMet-02}: \% of redundant code} \\ \hline
    \multicolumn{2}{|l|}{\textbf{EOSC-SCMet-10}: Number of comments} \\ \hline
    \multicolumn{2}{|l|}{\textbf{EOSC-Qual-27}: Intellectual Property} \\ \hline
    \multicolumn{2}{|l|}{\textbf{EOSC-SWRelMan-01}: Open source} \\ \hline
    \multicolumn{2}{|l|}{\textbf{EOSC-SWRelMan-02}: Version Control System (VCS)} \\ \hline
    \multicolumn{2}{|l|}{\textbf{EOSC-SWRelMan-10}: Open-source license} \\ \hline
    \multicolumn{2}{|l|}{\textbf{EOSC-SWRelMan-29}: Documentation online} \\ \hline
    \multicolumn{2}{|l|}{\textbf{EOSC-SWRelMan-30}: Documentation updates} \\ \hline
    \multicolumn{2}{|l|}{\textbf{EOSC-SWRelMan-32}: Documentation production} \\ \hline
    \multicolumn{2}{|l|}{\textbf{EOSC-SWTest-15}: No world-writable files or directories} \\ \hline

\end{supertabular}
\end{center}

\subsubsection{User story: Individual development}

The case of individual development of analysis script and workflows (table~\ref{tab:rs_recc01}), the recommendation are the common attributes listed in table~\ref{tab:rs_recc00}.

\begin{center}
    \tablehead{\hline \textbf{RS Stack} & \textbf{RS Type} \\ \hline}
    \tabletail{\hline}
    \bottomcaption{Recommended Quality Attributes, individual development, Analysis script and workflows.}
    \label{tab:rs_recc01}
    \small
    \begin{supertabular}{|p{0.65\linewidth}|p{0.35\linewidth}|} \hline
    4: \textbf{Project specific code} &
    4: \textbf{Analysis script and workflows} \\ \hline \hline
    \multicolumn{2}{|l|}{\textbf{All common Quality Attributes}: table \ref{tab:rs_recc00}} \\ \hline

\end{supertabular}
\end{center}

In the case of software type "Library", packaging and code deployment are added to the common attributes as listed in table \ref{tab:rs_recc02}.

\begin{center}
    \tablehead{\hline \textbf{RS Stack} & \textbf{RS Type} \\ \hline}
    \tabletail{\hline}
    \bottomcaption{Recommended Quality Attributes, individual development, library.}
    \label{tab:rs_recc02}
    \small
    \begin{supertabular}{|p{0.65\linewidth}|p{0.35\linewidth}|} \hline
    4: \textbf{Project specific code} &
    1: \textbf{Library} \\ \hline \hline

    \multicolumn{2}{|l|}{\textbf{All common Quality Attributes}: table \ref{tab:rs_recc00}} \\ \hline
    \multicolumn{2}{|l|}{\textbf{EOSC-SWRelMan-12}: Packaging} \\ \hline
    \multicolumn{2}{|l|}{\textbf{EOSC-SWRelMan-15}: Code deployment} \\ \hline

\end{supertabular}
\end{center}

\subsubsection{User story: Team Project}

The case for a team working on a project on analysis script and workflows, there is an increase in responsibility, as such it's recommended to add support that includes the reporting and solving bugs and source code hosting in common VCS repositories, as shown in table \ref{tab:rs_recc03}.

\begin{center}
    \tablehead{\hline \textbf{RS Stack} & \textbf{RS Type} \\ \hline}
    \tabletail{\hline}
    \bottomcaption{Recommended Quality Attributes, team project, analysis script and workflows.}
    \label{tab:rs_recc03}
    \small
    \begin{supertabular}{|p{0.65\linewidth}|p{0.35\linewidth}|} \hline
    4: \textbf{Project specific code} &
    4: \textbf{Analysis script and workflows} \\ \hline \hline

    \multicolumn{2}{|l|}{\textbf{All from User story: Individual development}: table \ref{tab:rs_recc01}} \\ \hline
    \multicolumn{2}{|l|}{\textbf{EOSC-TMet-02}: \# Resolved bugs} \\ \hline
    \multicolumn{2}{|l|}{\textbf{EOSC-TMet-03}: \# Open bugs} \\ \hline
    \multicolumn{2}{|l|}{\textbf{EOSC-SWRelMan-03}: Source code hosting} \\ \hline
    \multicolumn{2}{|l|}{\textbf{EOSC-SWRelMan-04}: Working state version} \\ \hline
    \multicolumn{2}{|l|}{\textbf{EOSC-SWRelMan-07}: Support} \\ \hline

\end{supertabular}
\end{center}

The case of research software type "Library", the recommendation is to add testing with respect to the previous cases, this is shown in table \ref{tab:rs_recc04}.

\begin{center}
    \tablehead{\hline \textbf{RS Stack} & \textbf{RS Type} \\ \hline}
    \tabletail{\hline}
    \bottomcaption{Recommended Quality Attributes, team project, library.}
    \label{tab:rs_recc04}
    \small
    \begin{supertabular}{|p{0.65\linewidth}|p{0.35\linewidth}|} \hline
    4: \textbf{Project specific code} &
    1: \textbf{Library} \\ \hline \hline

    \multicolumn{2}{|l|}{\textbf{All from User story: Individual development}: table \ref{tab:rs_recc02}} \\ \hline
    \multicolumn{2}{|l|}{\textbf{EOSC-TMet-02}: \# Resolved bugs} \\ \hline
    \multicolumn{2}{|l|}{\textbf{EOSC-TMet-03}: \# Open bugs} \\ \hline
    \multicolumn{2}{|l|}{\textbf{EOSC-SWRelMan-03}: Source code hosting} \\ \hline
    \multicolumn{2}{|l|}{\textbf{EOSC-SWRelMan-04}: Working state version} \\ \hline
    \multicolumn{2}{|l|}{\textbf{EOSC-SWRelMan-07}: Support} \\ \hline
    \multicolumn{2}{|l|}{\textbf{EOSC-SWTest-01}: Code style} \\ \hline
    \multicolumn{2}{|l|}{\textbf{EOSC-SWTest-02}: Unit tests} \\ \hline
    \multicolumn{2}{|l|}{\textbf{EOSC-SWTest-03}: Test doubles} \\ \hline
    \multicolumn{2}{|l|}{\textbf{EOSC-SWTest-07}: Functional testing} \\ \hline

\end{supertabular}
\end{center}

Table \ref{tab:rs_recc05} regards the development, deployment and operation of a service, added recommendations related to further types of tests such as APIs and integration (when applicable). Furthermore, there are several attributes related to securiry such as the use of service certificates to encrypt endpoints, use of strong authentication and authorization mechanisms even in the case of limited access to the services.

\begin{center}
    \tablehead{\hline \textbf{RS Stack} & \textbf{RS Type} \\ \hline}
    \tabletail{\hline}
    \bottomcaption{Recommended Quality Attributes, team project, services and platforms.}
    \label{tab:rs_recc05}
    \small
    \begin{supertabular}{|p{0.65\linewidth}|p{0.35\linewidth}|} \hline
    4: \textbf{Project specific code} &
    5: \textbf{Services and platforms} \\ \hline \hline

    \multicolumn{2}{|l|}{\textbf{All from User story: team project, library}: table \ref{tab:rs_recc04}} \\ \hline
    \multicolumn{2}{|l|}{\textbf{EOSC-SWTest-05}: API testing} \\ \hline
    \multicolumn{2}{|l|}{\textbf{EOSC-SWTest-06}: Integration testing} \\ \hline
    \multicolumn{2}{|l|}{\textbf{EOSC-SWTest-16}: Public endpoints and APIs encrypted} \\ \hline
    \multicolumn{2}{|l|}{\textbf{EOSC-SWTest-17}: Strong ciphers} \\ \hline
    \multicolumn{2}{|l|}{\textbf{EOSC-SWTest-18}: Authentication and Authorization} \\ \hline
    \multicolumn{2}{|l|}{\textbf{EOSC-SWTest-20}: Service compliance with data regulations (GDPR)} \\ \hline

\end{supertabular}
\end{center}

\subsubsection{User story: Team OSS}

This subsection describes the recommendations when a medium to large team is responsible for the development of Open Source Software for research. Thus, the increase in responsability with respect to the previous cases of smaller teams or individual researchers. Added Quality Attributes for more complete documentation, more types of tests including security and peer code review, as shown in table \ref{tab:rs_recc06}.

\begin{center}
    \tablehead{\hline \textbf{RS Stack} & \textbf{RS Type} \\ \hline}
    \tabletail{\hline}
    \bottomcaption{Recommended Quality Attributes, team OSS, library.}
    \label{tab:rs_recc06}
    \small
    \begin{supertabular}{|p{0.65\linewidth}|p{0.35\linewidth}|} \hline
    3 and 2: \textbf{Domain specific tools} and  \textbf{Scientific infrastructure} &
    1: \textbf{Library} \\ \hline \hline

    \multicolumn{2}{|l|}{\textbf{All from User story: Team Project}: table \ref{tab:rs_recc04}} \\ \hline
    \multicolumn{2}{|l|}{\textbf{EOSC-TMet-04}: Defect rates} \\ \hline
    \multicolumn{2}{|l|}{\textbf{EOSC-SWRelMan-08}: Code review} \\ \hline
    \multicolumn{2}{|l|}{\textbf{EOSC-SWRelMan-26}: Documentation version controlled} \\ \hline
    \multicolumn{2}{|l|}{\textbf{EOSC-SWRelMan-27}: Documentation as code} \\ \hline
    \multicolumn{2}{|l|}{\textbf{EOSC-SWRelMan-28}: Documentation formats} \\ \hline
    \multicolumn{2}{|l|}{\textbf{EOSC-SWTest-08}: Performance testing} \\ \hline
    \multicolumn{2}{|l|}{\textbf{EOSC-SWTest-13}: Static Application Security Testing (SAST)} \\ \hline
    \multicolumn{2}{|l|}{\textbf{EOSC-SWTest-14}: Security code reviews} \\ \hline

\end{supertabular}
\end{center}

Table \ref{tab:rs_recc07} show the case for a Framework or Application, it adds further types of testing such as scalability and development methodology called Test-Driven Development (TDD).

\begin{center}
    \tablehead{\hline \textbf{RS Stack} & \textbf{RS Type} \\ \hline}
    \tabletail{\hline}
    \bottomcaption{Recommended Quality Attributes, team OSS, frameworks and applications.}
    \label{tab:rs_recc07}
    \small
    \begin{supertabular}{|p{0.65\linewidth}|p{0.35\linewidth}|} \hline
    3 and 2: \textbf{Domain specific tools} and  \textbf{Scientific infrastructure} &
    2 and 3: \textbf{Framework, Application} \\ \hline \hline

    \multicolumn{2}{|l|}{\textbf{All from User story: Team OSS, Library}: table \ref{tab:rs_recc06}} \\ \hline
    \multicolumn{2}{|l|}{\textbf{EOSC-SWTest-04}: Test-Driven Development (TDD)} \\ \hline
    \multicolumn{2}{|l|}{\textbf{EOSC-SWTest-10}: Scalability testing} \\ \hline

\end{supertabular}
\end{center}

\subsubsection{User story: Team Service}

This subsection refers to a medium/large team developing an Open Source scientific research service or platform, that will be used by a large number of researchers, possibly from different fields. The list is significant with several types of tests including security, complete set of documentation, helpdesk for support and bug reporting.

\begin{center}
    \tablehead{\hline \textbf{RS Stack} & \textbf{RS Type} \\ \hline}
    \tabletail{\hline}
    \bottomcaption{Recommended Quality Attributes, team service, services and platforms.}
    \label{tab:rs_recc08}
    \small
    \begin{supertabular}{|p{0.65\linewidth}|p{0.35\linewidth}|} \hline
    3 and 2: \textbf{Domain specific tools} and \textbf{Scientific infrastructure} &
    5: \textbf{Services and platforms} \\ \hline \hline

    \multicolumn{2}{|l|}{\textbf{All from User story: Team Project}: table \ref{tab:rs_recc05}} \\ \hline
    \multicolumn{2}{|l|}{\textbf{EOSC-TMet-04}: Defect rates} \\ \hline
    \multicolumn{2}{|l|}{\textbf{EOSC-SWRelMan-08}: Code review} \\ \hline
    \multicolumn{2}{|l|}{\textbf{EOSC-SWRelMan-26}: Documentation version controlled} \\ \hline
    \multicolumn{2}{|l|}{\textbf{EOSC-SWRelMan-27}: Documentation as code} \\ \hline
    \multicolumn{2}{|l|}{\textbf{EOSC-SWRelMan-28}: Documentation formats} \\ \hline
    \multicolumn{2}{|l|}{\textbf{EOSC-SWTest-04}: Test-Driven Development (TDD)} \\ \hline
    \multicolumn{2}{|l|}{\textbf{EOSC-SWTest-08}: Performance testing} \\ \hline
    \multicolumn{2}{|l|}{\textbf{EOSC-SWTest-09}: Stress testing} \\ \hline
    \multicolumn{2}{|l|}{\textbf{EOSC-SWTest-10}: Scalability testing} \\ \hline
    \multicolumn{2}{|l|}{\textbf{EOSC-SWTest-12}: Open Web Application Security Project (OWASP)} \\ \hline
    \multicolumn{2}{|l|}{\textbf{EOSC-SWTest-13}: Static Application Security Testing (SAST)} \\ \hline
    \multicolumn{2}{|l|}{\textbf{EOSC-SWTest-14}: Security code reviews} \\ \hline
    \multicolumn{2}{|l|}{\textbf{EOSC-SWTest-19}: API security assessment} \\ \hline
    \multicolumn{2}{|l|}{\textbf{EOSC-SWTest-21}: Dynamic Application Security Testing (DAST)} \\ \hline
    \multicolumn{2}{|l|}{\textbf{EOSC-SWTest-22}: Interactive Application Security Testing (IAST)} \\ \hline
    \multicolumn{2}{|l|}{\textbf{EOSC-SWTest-23}: Security penetration testing} \\ \hline
    \multicolumn{2}{|l|}{\textbf{EOSC-SWTest-24}: Security assessment} \\ \hline
    \multicolumn{2}{|l|}{\textbf{EOSC-SWTest-25}: Security as Code (SaC) Testing} \\ \hline

\end{supertabular}
\end{center}


\subsection{Example of tools, services and infrastructures to implement Quality Assurance for RS}

% \violaine{The idea is to refer to some search software of the different categories and show the similarities and differences with
% the analysis of the top to point out the possible lacks in the current service offer} 

\subsubsection{About version control}

The quality attributes concerned by version control are the following :
\begin{itemize}
  \item EOSC-SWRelMan-02: Version Control System (VCS)
  \item EOSC-SWRelMan-03: Source code hosting
  \item EOSC-SWRelMan-04: Working state version
\end{itemize}

Version control is the essential basic tool for software development.
Software forges have become vital tools for all software developers.
This type of infrastructures is often deployed, at different level :
laboratory, project, institution. Commercial software forge (as
github.com, gitlab.com ...) are also significantly used in academic
context. A recent report of the Software and
Sources Codes College of the Committee for Open Science illustrate the
landscaping in France \cite{leberre:hal-04208924}.

An important point is that most of the software forge integrate a huge
variety of tools, most of them listed below, and not only control version.

\subsubsection{About intellectual property and license}

The quality attributes concerned by legal issues are the following :
\begin{itemize}
  \item EOSC-Qual-27: Intellectual Property
  \item EOSC-SWRelMan-01: Open source
  \item EOSC-SWRelMan-10: Open-source license
\end{itemize}

There is no real tools or infrastructure for the management of legal
issues, the most important is here the support provided by he
institutions.

This support must be as large as possible including the help about
Developer Certificate of Origin (DCO), or Contributor License
Agreement (CLA).

\subsubsection{About packaging}

\begin{itemize}
  \item EOSC-SWRelMan-12: Packaging
  \item EOSC-SWRelMan-15: Code deployment
\end{itemize}

guix, nix, containers ...

\subsubsection{About documentation}

\begin{itemize}
  \item EOSC-SWRelMan-29: Documentation online
  \item EOSC-SWRelMan-30: Documentation updates
  \item EOSC-SWRelMan-32: Documentation production
  \item EOSC-SWRelMan-26: Documentation version controlled
  \item EOSC-SWRelMan-27: Documentation as code
  \item EOSC-SWRelMan-28: Documentation formats
\end{itemize}

automatic generation
on line doc, tutorials ...


\subsubsection{About bugs management}

\begin{itemize}
  \item EOSC-TMet-02: \# Resolved bugs
  \item EOSC-TMet-03: \# Open bugs
\end{itemize}

bug tracking system

\subsubsection{About tests}

\begin{itemize}
  \item EOSC-SWTest-02: Unit tests
  \item EOSC-SWTest-03: Test doubles
  \item EOSC-SWTest-07: Functional testing
  \item EOSC-SWTest-05: API testing
  \item EOSC-SWTest-06: Integration testing
  \item EOSC-SWTest-08: Performance testing
  \item EOSC-SWTest-13: Static Application Security Testing (SAST)
  \item EOSC-SWTest-04: Test-Driven Development (TDD)
  \item EOSC-SWTest-10: Scalability testing
  \item EOSC-SWTest-09: Stress testing
\end{itemize}

Tests management
CI


\subsubsection{About support}

\begin{itemize}
  \item EOSC-SWRelMan-07: Support
\end{itemize}

forums
lists

\subsubsection{About code analysis}

\begin{itemize}
  \item EOSC-SCMet-02: \% of redundant code
  \item EOSC-SCMet-10: Number of comments
  \item EOSC-SWTest-15: No world-writable files or directories
  \item EOSC-SWTest-01: Code style
  \item EOSC-SWRelMan-08: Code review
\end{itemize}

Programmation rules
static analysis
coverage
code review

\subsubsection{About security}

\begin{itemize}
  \item EOSC-SWTest-16: Public endpoints and APIs encrypted
  \item EOSC-SWTest-17: Strong ciphers
  \item EOSC-SWTest-18: Authentication and Authorization
  \item EOSC-SWTest-20: Service compliance with data regulations (GDPR)
  \item EOSC-TMet-04: Defect rates
  \item EOSC-SWTest-14: Security code reviews
  \item EOSC-SWTest-12: Open Web Application Security Project (OWASP)
  \item EOSC-SWTest-13: Static Application Security Testing (SAST)
  \item EOSC-SWTest-14: Security code reviews
  \item EOSC-SWTest-19: API security assessment
  \item EOSC-SWTest-21: Dynamic Application Security Testing (DAST)
  \item EOSC-SWTest-22: Interactive Application Security Testing (IAST)
  \item EOSC-SWTest-23: Security penetration testing
  \item EOSC-SWTest-24: Security assessment
  \item EOSC-SWTest-25: Security as Code (SaC) Testing
\end{itemize}

\subsubsection{Tranversal needs}

Training

At different levels (students, researchers, engineers, legal services ...)

