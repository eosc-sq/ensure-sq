\subsection{Research Software stack and types}

Research Software stack is taken from~\cite{hinsen2019}:

\begin{center}
    \tablehead{
        \hline
        \textbf{Stack} & \textbf{Stack definition} & \textbf{Types} \\
        \hline
    }
    \tabletail{\hline}
    \tablelasttail{\hline}
    
    \bottomcaption{RS stacks, stacks definitions and RS types. Stack level 1 is not considered in this report.}
    \label{tab:rs_stacks}

    \small
    \begin{supertabular}{|p{0.15\linewidth}|p{0.5\linewidth}|p{0.3\linewidth}|}

    \hline
    4 - Project specific code &
    Software written by scientists for a specific research project. It can take various forms including scripts, notebooks, and workflows, but also special-purpose libraries and utilities. &
    Library; Analysis script, workflows; Services and platforms \\ \hline

    3 - Domain specific tools &
    Tools and libraries that implement models and methods which are developed and used by specific communities. Gromacs, MMTK, Amber. &
    Library; Application (such as Monte-Carlo simulation); Services and platforms
    \\ \hline

    2 - Scientific infrastructure &
    Infrastructure created specifically for scientific computing, but not for any particular application domain: mathematical libraries such as BLAS, LAPACK, or SciPy, scientific data management tools such as HDF5. &
    Library; Framework; Services and platforms\\ \hline

    \it{1 - Non-scientific infrastructure} &
    \it{Compilers and interpreters, libraries for data management; gcc, python...} &
    -- \\ \hline

    \end{supertabular}
\end{center}

The RS types can be summarised next:

\begin{itemize}
\item Library
\item Framework
\item Application (such as Monte-Carlo simulation)
\item Analysis script
\item Services and platforms
\end{itemize}

\subsection{Definition and references of Research SW}

% \dg{[summary discussion] The expectation of research software is that it can support the findings in a paper. In that sense, RS is "expected" to
% reproduce those findings, while regular software would not necessarily have this expectation (as long as it works).}

The goal of this section is to cross-reference the elements of
software quality from the previous section with the SG1 typology of
research software development, in order to identify the tools and
infrastructures needed to satisfy minimum quality criteria. 


As a reminder, SG1 has defined different categories of software
development related to the context in which they are performed. Others have different categories as well, for instance \url{https://docs.google.com/presentation/d/1uwxSwd8chbG7bVn5lPvNNhv5f_JeeelA2mN9NXhlOhA/edit#slide=id.g183ddf12e8b_0_122} and \url{https://zenodo.org/record/7589725#.ZGc5E3bP2Uk}
