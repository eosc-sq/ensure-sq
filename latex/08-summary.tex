\miguel{I'm working on this\dots}

In this report we have tried to provide insights on how the quality of research software could be improved from several different points of view, as well as providing practical recommendations.

We avoided that the professional views of the authors contributing to this document affected its objectivity. Indeed, each of us have had our own experience when operating software and services, which would not necessarily be representative or general enough for all users and situations.

To overcome this problem, we conducted a systematic and thorough survey of the existing literature according to the keywords, title, and field of the articles. The list was further processed to exclude non-suitable articles (for example  those not actually discussing about quality aspects, among other criteria). Finally, we ended up with a set of relevant articles that were included in our study, that led to a classification into software quality characteristics and software quality attributes and metrics. Given that the classification is motivated from what is found in significant published peer-reviewed articles, they provide an unbiased list of characteristics, that we further classified into significant attributes and metrics. We summarised the list of attributed and metrics in Appendix~\ref{appendix_qa}, where each entry contains our own proposed codename, their name in the source reference, the associated characteristics, its definition, its context of use, and a reference to the source article to help tracking.

The survey of the broad existing literature and the classification took a large time and effort, but we are convinced that it was the only proper way to proceed in order to describe the landscape of research software and, eventually, propose recommendations.

In the Landscaping section of the article we tried first to define what research software. Our study is specific for research software and therefore it was needed to give a definition. We realised that this attempt could probably end up in an endless discussion before reaching an agreement on a precise definition. Therefore, we defined research software as any software which is linked to a scientific publication, given that this is surely the most important of its characteristics. 

Software systems can be, in general, organised in different ways, one of them being a stack of components and their dependencies. We reviewed the stack on which software is built upon, from libraries to complete services and platforms. 

To complete the landscape section, we included a discussion about what could be the expectations depending on the type of research software and depending on specific layers of the software stack. From the first classification one can clearly notice that the requirements for a team to build a complete services and very different from those of individual researchers who builds their own scripts to produce results in a publication, for example. In the second classification we discussed on the different contexts where software could be considered, as for example a scientific structure or a domain-specific tool.

The classification on characteristics, attributes, and metrics from the survey, along with the landscaping on research software, allowed us to propose some practical recommendations.